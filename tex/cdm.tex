\documentclass[a4paper]{article}
\usepackage{pgfplotstable}
\usepackage{longtable}
\usepackage{varwidth}
\usepackage{array}
\usepackage{lscape}
\usepackage[top=2 cm,bottom=2cm]{geometry}
\usepackage{helvet}
\usepackage{tabto}
\renewcommand{\familydefault}{\sfdefault}


\title {Copernicus Climate Change Service - 311a Lot 2 \\ Defining a Common Data Model}
\author {David I. Berry \\ National Oceanography Centre, UK}
\date {28th June 2017}


\pgfplotstableset{
empty header/.style={
           typeset cell/.append code={%
               \ifnum\pgfplotstablerow=-1 %
                   \pgfkeyssetvalue{/pgfplots/table/@cell content}{}%
               \fi
           }
       }
   }


\setlength{\parindent}{0pt}

\begin{document}


\maketitle

\vskip 0.25in
\section*{Summary}
\hrule
\vskip 0.25in
\noindent This document describes background information and a summary of initial steps taken towards defining a common data model for the representation of in situ observations as part of the C3S 311a activity.\\ \\
\noindent An overview of the preferred data model from Lot 2 is given and participants on the call are invited to:\\
\begin{itemize}
\item Review the background information and proposed data model presented in this document
\item Endorse the proposed data model or propose an alternative model for use within C3S 311a.
\end{itemize}
\vskip 0.25in
\hrule

\newpage
\tableofcontents
\newpage
\listoftables
\newpage


\section {Introduction}

The Copernicus Climate Change Service (C3S), through its "Collection and Processing of In Situ Observations (C3S 311a)" tender, seeks to harmonise both data formats and metadata (discovery and observational) conventions. The first step of this process, as noted within the invitation to tender, is the development or adoption of a common data model\footnote{From the ITT: A common data model is different from a file format, which defines how information is encoded in a file. The purpose of a data model is to provide a well-defined data structure that can be used to represent data records from a variety of sources, in such a way that the information contained in those records can be unambiguously accessed using a common set of tools. Development of a common data model for observations involves specification of data attributes and their symbolic names, including, for example, identifiers for different instruments, observed parameters, geolocation and timing, etc. A governance structure is required to manage such specifications, ensure consistency with standards where they exist, and to ensure a controlled evolution of the data model.} for the data and metadata. Within this document, when complete, we will describe the common data model developed within Lot(s) 1 - 4 of the C3S 311a tender in consultation with ECMWF. The themes for the Lots 1 - 4 are:\\
\begin{itemize}
\item Lot 1 - Coordination of data rescue activities
\item Lot 2 - Access to observations from global climate data archives
\item Lot 3 - Access to observations from baseline and reference networks
\item Lot 4 - Climate monitoring products for Europe based on in situ observations.
\end{itemize}
\textbf{Lot 1 (C3S DRS)} are building a new data portal, led by the WMO I-DARE portal lead from KNMI, that will be melded together with a much enhanced EU FP 7 ERA-CLIM 2 data registry, led by that project's Portuguese lead, plus new and enhanced data tools and techniques led by the University of Bern group. Data rescue accounts for only 10 - 15\% of the Lot 1 budget, and is focused on three regions in the Southern Hemisphere in and around Argentina, South Africa and in the New Zealand to Drake Passage sector, but will link closely to the larger data rescue efforts of ACRE, IEDRO, ICA\&D and similar. As with Lot 2, Lot 1 will deal with the full range of historical terrestrial and marine surface weather observations plus upper air data, serving the various international repositories these data are held in, plus having the capacity to deal with their metadata (including a compendium of all data forms/templates these data are recorded on), scanned images of hard copy data, and weather and analogue (pluviograms, thermograms, barograms etc) charts etc.\\

\textbf{Within Lot 2}, observations and metadata from land stations and marine platforms will be harmonised into a common data model and a web based service developed to serve the data through the C3S Climate Data Store (CDS). The observations include instantaneous / point observations, such as those from SYNOP weather reports, as well as daily and monthly summaries (CLIMAT DAILY and CLIMAT). A single report may contain observations of multiple parameters, e.g. air temperature, humidity, wind speed etc. The stations range from stationary land stations to mobile merchant ships, drifting buoys and other marine platforms.\\

\textbf{Lot 3} are creating a harmonized observational dataset of measurements from the Global Baseline and Reference radiosounding networks. Within the first year observations are restricted to temperature and humidity measurements, in future years this will be expanded to include other essential climate variables (surface temperature, wind, ozone, trace gases, GPS IWV). Observations from the GRUAN and GUAN networks will be the main focus, but with potential extension to the broader RAOB program. Annual updates will be provided. Integrated physical and statistical corrections will be used to improve the quality of the baseline observations using the data from the reference networks. Lot 3 intend to be fully compliant with ODB version 2 (ODB2), noting that some changes will be necessary to ODB2 to report the full range of information required. Discovery metadata are planned to be compliant with ISO 19115 and observational metadata reported using the CF conventions. Compliance with the WIGOS metadata standard is also expected. \\

\textbf{Lot 4} will build on and extend the European Climate Assessment and Dataset (ECA\&D) project and E-OBS daily dataset for Europe. The gridded E-OBS dataset was initially developed as part of the ENSEMBLES project for statistical comparisons with Regional Climate Model output (Haylock et al., 2008). More recently European research projects EURO4M, UERRA, EUPORIAS, EUSTACE, and CLIPc led to further improvements and applications, and ECA\&D/E-OBS has now become reference datasets for a larger user community, also outside climate research. Funding by EUMETNET and KNMI supported the developments of additional functionality, and the close collaboration with EUMETNET members has led to strongly improved ECA\&D station coverage over Europe in recent years. Within C3S\_311a lot 4, the ECA\&D and E-OBS will be transformed into an operational system for the Copernicus Climate Change Service (C3S), delivering regularly updated gridded products based on European in-situ data for many Essential Climate Variables (ECVs). The underlying station data that include surface air temperature, precipitation, humidity, wind speed and direction, will be made available as well, pending permission by the owners of these data. To serve climate change monitoring and climate impact assessments a large number of user-oriented climate indices will be provided, both as time series at station sites and as gridded products. No preference has been specified for the data models to be used.\\


Section 2 of this report provides background information on joint activities between Lots 2 and 3 so far, the ECMWF Observations DataBase (ODB) data model and relevant WMO data models. Section 3 gives an overview of the preferred data model from Lot 2 and proposes a list of elements for the observations table. Auxiliary tables are also proposed in Section 3 but left empty for future discussion once the principles of the type of data model have been agreed across lots. Section 4 proposes a governance mechanism for the common data model across lots and next steps required.\\


\section {Background and existing standards}
\subsection {ODB and tenders for Lots 2 and 3}

Both Lots 2 and 3 have proposed using data models based on the data model developed by ECMWF as part of the Observations DataBase (ODB) software. Within the ODB type data model each observation of a single parameter is stored as a separate record, with a single report spanning multiple records. Within each record the station / report information is repeated. A simplified example is shown in Table 1.\\

\pgfplotstabletypeset[
   empty header,
    begin table=\begin{longtable},
    %every head row/.style={output empty row},
    every nth row={1}{before row=\hline},
    every first row/.append style={
        before row={%
            % Initial caption
            \caption{Simplified example of records in ODB type data model, with observations from reports 1 and 2 spanning multiple records. For simplicity, the z coordinate has been omitted but profile data would be represented with each layer / height as a separate record}
            \label{tab:DataTableAdjustment}\\
            % Initial column headers
            \hline\hline
            \multicolumn{5} {c} { \textbf{header information}} & 
            \multicolumn{3} {c} { \textbf{observation information}} 
            \\
            \hline
            \multicolumn{1} { > {\centering}V{0.3 in}} { \textbf{record id}} & 
            \multicolumn{1} { > {\centering}V{0.3 in}} { \textbf{report id}} & 
            \multicolumn{1} { > {\centering}V{0.3 in}} { \textbf{obs id}} & 
            \multicolumn{1} { > {\centering}V{1 in}} { \textbf{date}} & 
            \multicolumn{1} { > {\centering}V{1 in}} { \textbf{location}} & 
            \multicolumn{1} { > {\centering}V{0.75 in}} { \textbf{parameter}} & 
            \multicolumn{1} { > {\centering} V{0.75 in} } {\textbf{value}} &
            \multicolumn{1} { > {\centering} V{0.75 in} } {\textbf{units}} 
            \\ \hline\hline \endfirsthead
            \multicolumn{6}{c}{Table \thetable\ adjustment (cont.)} \\
            % column headers on additional pages
            \hline\hline 
            \multicolumn{5} { > {\centering}V{2.9 in}} { \textbf{header information}} & 
            \multicolumn{3} { > {\centering}V{2.25 in}} { \textbf{observation information}} 
            \\
            \hline
            \multicolumn{1} { > {\centering}V{0.3 in}} { \textbf{record id}} & 
            \multicolumn{1} { > {\centering}V{0.3 in}} { \textbf{report id}} & 
            \multicolumn{1} { > {\centering}V{0.3 in}} { \textbf{obs id}} & 
            \multicolumn{1} { > {\centering}V{1 in}} { \textbf{date}} & 
            \multicolumn{1} { > {\centering}V{1 in}} { \textbf{location}} & 
            \multicolumn{1} { > {\centering}V{0.75 in}} { \textbf{parameter}} & 
            \multicolumn{1} { > {\centering} V{0.75 in} } {\textbf{value}} &
            \multicolumn{1} { > {\centering} V{0.75 in} } {\textbf{units}} 
            \\ \hline\hline \endhead
            % Footer on 1st to penultimate pages
            \multicolumn{6}{r}{{Continued on next page}} \\
            \endfoot
            % Footer on last page of table
            \hline
            \multicolumn{6}{r}{{End of table}} \\ 
            \endlastfoot
        }
    },
    %
    end table=\end{longtable},
    col sep = &,
    row sep = \\ , 
    columns/record id/.style={string type, column type = V{0.3 in}},
    columns/report id/.style={string type, column type = V{0.3 in}},
    columns/observation id/.style={string type, column type = V{0.3 in}},
    columns/date/.style={string type, column type = V{1 in}},
    columns/location/.style={string type, column type = V{1 in}}, 
    columns/parameter/.style={string type, column type = V{1 in}},
    columns/value/.style={string type, column type = V{0.3 in}},
    columns/units/.style={string type, column type = V{0.3 in}}
]{
record id& report id& observation id& date & location & parameter & value & units \\
1 & 1 & 1 & 2012-01-01 12:00+0.0 & POINT(-40 40) & air temperature & 300.0 & K \\
2 & 1 & 2 & 2012-01-01 12:00+0.0 & POINT(-40 40) & sea level pressure & 1013.0 & hPa \\
3 & 2 & 3 & 2012-01-01 18:00+0.0 & POINT(-40.1 40.2) & air temperature & 300.3 & K \\
4 & 2 & 4 & 2012-01-01 18:00+0.0 & POINT(-40.1 40.2) & sea level pressure & 1013.2 & hPa \\
}

The implementation of the ODB model at ECMWF, that proposed in Lots 2 and 3 all have differing requirements. For example, the existing observations table columns defined within ODB\footnote{http://apps.ecmwf.int/odbgov/column/}  contain many parameters that are of little relevance to the In Situ observations but are relevant to the assimilation of data from many different sources into the numerical models. Conversely, there are many parameters included in the data from Lots 2 and 3 that are required to correctly interpret the observations but that are not included in ODB. \\

In order to facilitate the development of the data model there have been two initial teleconferences between Lots 2 and 3 discussing the CDM and collating information on the parameters required . Each parameter and report type has its own unique set of fields and metadata fields. For example, surface air temperature observations are typically made in a screen or shelter that can influence the quality of the measurements. As a result, it is desirable to include information on the screen type, material and dimensions alongside the observation. For upper air temperature observations this metadata information is not relevant but other parameters will be required, such as the type of balloons used, instrument type and burstpoint.\\

In order to represent the wide variety of metadata required across (and within) Lots three different solutions are possible:\\
\begin{itemize}
\item The observations table is expanded to include all possible metadata fields, with new columns added when a new data / report type is included.
\item Each report (and possibly parameter) type has a separate observations table, with a minimum set of common parameters defined across the different tables.
\item The observations table is defined to include the minimum set of information required for each observation and the metadata is then linked via a series of Entity-Attribute-Value (EAV) based tables (e.g. see Table 2).
\end{itemize}

Within this document we are proposing to use solution (3), defining a minimum set of parameters to be included in the observations table and linking to the metadata in auxiliary tables. Solution (1) has been discounted as being impractical from an implementation perspective and from the perspective of adding new data types at a future date. Option (2) has not been discounted but will result in a series of data models being defined rather than a single unified data model.\\

<<<<<<< HEAD
\pgfplotstabletypeset[
   empty header,
    begin table=\begin{longtable},
    %every head row/.style={output empty row},
    every nth row={1}{before row=\hline},
    every first row/.append style={
        before row={%
            % Initial caption
            \caption{Simplified  example for EAV type table for profile (atmospheric and oceanic) data.}
            \label{tab:DataTableEAV}\\
            % Initial column headers
            \hline\hline
            \multicolumn{1} { > {\centering}V{0.5 in}} { \textbf{report id}} & 
            \multicolumn{1} { > {\centering}V{0.5 in}} { \textbf{report type}} & 
            \multicolumn{1} { > {\centering}V{1.5 in}} { \textbf{field}} & 
            \multicolumn{1} { > {\centering}V{1 in}} { \textbf{value coded}} & 
            \multicolumn{1} { > {\centering}V{1 in}} { \textbf{value numeric}}
            \\ \hline\hline \endfirsthead
            \multicolumn{5}{c}{Table \thetable\ adjustment (cont.)} \\
            % column headers on additional pages
            \hline\hline 
            \multicolumn{1} { > {\centering}V{0.5 in}} { \textbf{report id}} & 
            \multicolumn{1} { > {\centering}V{0.5 in}} { \textbf{report type}} & 
            \multicolumn{1} { > {\centering}V{1.5 in}} { \textbf{field}} & 
            \multicolumn{1} { > {\centering}V{1 in}} { \textbf{value coded}} & 
            \multicolumn{1} { > {\centering}V{1 in}} { \textbf{value numeric}}
            \\ \hline\hline \endhead
            % Footer on 1st to penultimate pages
            \multicolumn{5}{r}{{Continued on next page}} \\
            \endfoot
            % Footer on last page of table
            \hline
            \multicolumn{5}{r}{{End of table}} \\ 
            \endlastfoot
        }
    },
    %
    end table=\end{longtable},
    col sep = &,
    row sep = \\ , 
    columns/report id/.style={string type, column type = V{0.5 in}},
    columns/report type/.style={string type, column type = V{0.5 in}},
    columns/field/.style={string type, column type = V{1.5 in}},
    columns/value coded/.style={string type, column type = V{1 in}}, 
    columns/value numeric/.style={string type, column type = V{1 in}}
]{
report id & report type & field & value coded & value numeric \\
4 & GRUAN & Ascent Balloon Number & 1 & NA \\
4 & GRUAN & Ascent Balloon Type & 1 & NA \\
4 & GRUAN & Ascent balloon weight (g) & NA & 100.0 \\
}
=======
% \input{example_eav_table}
Example EAV table here ... \\
>>>>>>> origin/master

\subsection {BUFR and WIGOS Metadata Standard}
Prior to defining the data model it is useful to refer to both the WMO Binary Universal Form for the Representation of meteorological data (BUFR) (WMO, 2015a) and the WMO Integrated Observing System Metadata Standard (WMDS) (WMO, 2015b). \\

The BUFR format is a flexible and efficient table driven format for reporting weather observations on the WMO Global Telecommunications System (GTS) in binary. The tables defined as part of the BUFR format include many of the parameters that will be included in the CDM. For example, Common code table C6 (WMO 2015a) includes all the measurement units reportable in BUFR (and other WMO codes). Similarly, code tables are defined for reporting instrument types and methods, station types etc. Where possible, these code tables should be referenced and used in preference to defining new code tables.\\

In recognition of the increasing importance of observational metadata the WMDS is currently under development and undergoing a phased implementation (WMO, 2015b). The WMDS forms an extension of the ISO19115 metadata standard, with additional mandatory elements describing both the station level and discovery metadata as well as specific information on the instrumentation used and processing steps. As part of the process simplified versions of BUFR and other tables have been included in the standard. As with BUFR these tables should be referenced, where appropriate, in preference to defining new code tables. Additionally, for compatibility with WIGOS the CDM should contain all mandatory elements of the WMDS. 

\section {Common Data Model}

As noted above, we are proposing a data model based on the ODB type data model, but with the metadata linked through a series of auxiliary / configuration tables. A schematic of this is shown in Figure 1. The observations table is described fully below (Table 3) and contains the geospatial (xyz) and temporal (t) locations of both the station making the report and the observed parameter, unique identifying information for the station, source data (i.e. dataset) information, observed values and data licencing / usage rights. In Table 3 below, where we list the proposed elements for the observations table, we also identify where there is overlap with the elements required by the WMDS. It should be noted that not all elements from the WMDS will appear in the observations table but will be included in the auxiliary tables. \\

Simplified CDM schematic here\\

To enable flexibility and accommodate the diverse data types and metadata the additional tables are proposed to be EAV based (see Table 2 above for example). This also gives the flexibility of adding a new metadata field by simply adding a new row rather than column. Within the following tables the following syntax has been used to indicate the data type for the different elements: \\
\begin{itemize}
\item numeric: \tabto{3 cm} Any numeric value (integer or floating point).
\item int: \tabto{3 cm} An integer value.
\item varchar: \tabto{3 cm} A variable length character string.
\item timestamp: \tabto{3 cm} A timestamp, e.g. "2017-07-01 00:00:0.0+00".
\item {[]}:\tabto{3 cm} An array of the indicated type.
\item (fk)  \tabto{3 cm} The indicated value is also a foreign key linking to another table.
\end {itemize}

%\input{cdm_schematic}


%\begin{landscape}

\subsection {Observations table}

Preamble text ... \\

\begin{landscape}
\pgfplotstabletypeset[
    empty header,
    begin table=\begin{longtable},
    %every head row/.style={output empty row},
    every nth row={1}{before row=\hline},
    every first row/.append style={
        before row={%
            % Initial caption
            \caption{observations\_table}
            \label{tab:DataTableObservationstable}\\
            % Initial column headers
            \hline\hline \multicolumn{1} { > {\centering}V{1.340000 in}} { \textbf{element\_number}} & 
\multicolumn{1} { > {\centering}V{1.340000 in}} { \textbf{element\_name}} & 
\multicolumn{1} { > {\centering}V{1.340000 in}} { \textbf{kind}} & 
\multicolumn{1} { > {\centering}V{1.340000 in}} { \textbf{external\_table}} & 
\multicolumn{1} { > {\centering}V{1.340000 in}} { \textbf{wigos}} & 
 \multicolumn{1} { > {\centering} V{3.000000 in} } {\textbf{description}} \\ \hline\hline \endfirsthead
            \multicolumn{6}{c}{Table \thetable\ observations\_table (cont.)} \\
            % column headers on additional pages
            \hline\hline \multicolumn{1} { > {\centering}V{1.340000 in} } { \textbf{element\_number}} & 
\multicolumn{1} { > {\centering}V{1.340000 in} } { \textbf{element\_name}} & 
\multicolumn{1} { > {\centering}V{1.340000 in} } { \textbf{kind}} & 
\multicolumn{1} { > {\centering}V{1.340000 in} } { \textbf{external\_table}} & 
\multicolumn{1} { > {\centering}V{1.340000 in} } { \textbf{wigos}} & 
 \multicolumn{1} { > {\centering} V{3.000000 in} } {\textbf{description}} \\ \hline\hline \endhead
            % Footer on 1st to penultimate pages
            \multicolumn{6}{r}{{Continued on next page}} \\
            \endfoot
            % Footer on last page of table
            \hline
            \multicolumn{6}{r}{{End of table}} \\ 
            \endlastfoot
        }
    },
    %
    end table=\end{longtable},
    col sep=tab,
    string type,
    columns/element_number/.style={
            string type, 
            column type= V{1.340000 in}, 
            string replace*={_}{\_}
        },
    columns/element_name/.style={
            string type, 
            column type= V{1.340000 in}, 
            string replace*={_}{\_}
        },
    columns/kind/.style={
            string type, 
            column type= V{1.340000 in}, 
            string replace*={_}{\_}
        },
    columns/external_table/.style={
            string type, 
            column type= V{1.340000 in}, 
            string replace*={_}{\_}
        },
    columns/wigos/.style={
            string type, 
            column type= V{1.340000 in}, 
            string replace*={_}{\_}
        },
    columns/description/.style={
            string type, 
            string replace*={_}{\_},
            column type = V{3.000000 in}
        }
    ]{/Users/dyb/GitHub/C3S_311a_CDM/tables/observations_table.csv}
\end{landscape}


\subsection {Station configuration table}

%Entity-attribute value based table for station configuration (and others).
\documentclass{article}

\usepackage[a4paper, total={6in, 9.5in}]{geometry}
\usepackage{pgfplotstable} 
\usepackage{longtable}
\usepackage{booktabs}
\renewcommand{\familydefault}{\sfdefault}




\pgfplotstableset{
begin table=\begin{longtable},
end table=\end{longtable},
}


\title {Copernicus Climate Change Service - 311a Lot 2\\Defining a common data model}
\author {David I. Berry}
\date {23 June 2017}
\begin{document}
\maketitle
\section {Decode tables}

\begin{table}[h!]
  \begin{center}
    \caption{Station configuration}
    \pgfplotstabletypeset[
        col sep=tab, string type,
        header=true,
        every head row/.style={before row=\toprule\toprule, after row=\bottomrule\bottomrule\endhead}, 
        every last row/.style={after row=\bottomrule},
        after row=\hline,
    ]{/Users/dyb/Documents/Projects/C3S_311a_Lot2/WP2/CDM/github/tables/station_configuration.csv}
  \end{center}
\end{table}

\end{document}



\subsection {Source configuration table}

\begin{landscape}
\pgfplotstabletypeset[
    empty header,
    begin table=\begin{longtable},
    %every head row/.style={output empty row},
    every nth row={1}{before row=\hline},
    every first row/.append style={
        before row={%
            % Initial caption
            \caption{source\_configuration}
            \label{tab:DataTableSourceconfiguration}\\
            % Initial column headers
            \hline\hline             \multicolumn{1} { V{1.475000 in}} { \textbf{\seqsplit{element\_number}}} & 
            \multicolumn{1} { V{1.475000 in}} { \textbf{\seqsplit{element\_name}}} & 
            \multicolumn{1} { V{1.475000 in}} { \textbf{\seqsplit{type}}} & 
            \multicolumn{1} { V{1.475000 in}} { \textbf{\seqsplit{external\_table}}} & 
            \multicolumn{1} { V{3.000000 in} } {\textbf{\seqsplit{description}}} \\ \hline\hline \endfirsthead
            \multicolumn{5}{c}{Table \thetable\ source\_configuration (cont.)} \\
            % column headers on additional pages
            \hline\hline             \multicolumn{1} {V{1.475000 in} } { \textbf{\seqsplit{element\_number}}} & 
            \multicolumn{1} {V{1.475000 in} } { \textbf{\seqsplit{element\_name}}} & 
            \multicolumn{1} {V{1.475000 in} } { \textbf{\seqsplit{type}}} & 
            \multicolumn{1} {V{1.475000 in} } { \textbf{\seqsplit{external\_table}}} & 
            \multicolumn{1} { V{3.000000 in} } {\textbf{\seqsplit{description}}} \\ \hline\hline \endhead
            % Footer on 1st to penultimate pages
            \multicolumn{5}{r}{{Continued on next page}} \\
            \endfoot
            % Footer on last page of table
            \hline
            \multicolumn{5}{r}{{End of table}} \\ 
            \endlastfoot
        }
    },
    %
    end table=\end{longtable},
    col sep=tab,
    string type,
    columns/element_number/.style={
            string type, 
            column type= V{1.475000 in}, 
            string replace*={_}{\_}
        },
    columns/element_name/.style={
            string type, 
            column type= n{1.475000 in}, 
            string replace*={_}{\_}
        },
    columns/type/.style={
            string type, 
            column type= V{1.475000 in}, 
            string replace*={_}{\_}
        },
    columns/external_table/.style={
            string type, 
            column type= n{1.475000 in}, 
            string replace*={_}{\_}
        },
    columns/description/.style={
            string type, 
            string replace*={_}{\_},
            column type = V{3.000000 in}
        }
    ]{/Users/dyb/GitHub/C3S_311a_CDM/tables/source_configuration.csv}
\end{landscape}


\subsection {Profile configuration table}

\pgfplotstabletypeset[
    empty header,
    begin table=\begin{longtable},
    %every head row/.style={output empty row},
    every nth row={1}{before row=\hline},
    every first row/.append style={
        before row={%
            % Initial caption
            \caption{Profile configuration}
            \label{tab:DataTable}\\
            % Initial column headers


\subsection {Sensor configuration table}

\begin{landscape}
\subsection {Sensor configuration table}
\pgfplotstabletypeset[
    empty header,
    begin table=\begin{longtable},
    %every head row/.style={output empty row},
    every nth row={1}{before row=\hline},
    every first row/.append style={
        before row={%
            % Initial caption
            \caption{sensor\_configuration}
            \label{tab:DataTableSensorconfiguration}\\
            % Initial column headers
            \hline\hline             \multicolumn{1} { V{1.475000 in}} { \textbf{\seqsplit{element\_number}}} & 
            \multicolumn{1} { V{1.475000 in}} { \textbf{\seqsplit{element\_name}}} & 
            \multicolumn{1} { V{1.475000 in}} { \textbf{\seqsplit{type}}} & 
            \multicolumn{1} { V{1.475000 in}} { \textbf{\seqsplit{external\_table}}} & 
            \multicolumn{1} { V{3.000000 in} } {\textbf{\seqsplit{description}}} \\ \hline\hline \endfirsthead
            \multicolumn{5}{c}{Table \thetable\ sensor\_configuration (cont.)} \\
            % column headers on additional pages
            \hline\hline             \multicolumn{1} {V{1.475000 in} } { \textbf{\seqsplit{element\_number}}} & 
            \multicolumn{1} {V{1.475000 in} } { \textbf{\seqsplit{element\_name}}} & 
            \multicolumn{1} {V{1.475000 in} } { \textbf{\seqsplit{type}}} & 
            \multicolumn{1} {V{1.475000 in} } { \textbf{\seqsplit{external\_table}}} & 
            \multicolumn{1} { V{3.000000 in} } {\textbf{\seqsplit{description}}} \\ \hline\hline \endhead
            % Footer on 1st to penultimate pages
            \multicolumn{5}{r}{{Continued on next page}} \\
            \endfoot
            % Footer on last page of table
            \hline
            \multicolumn{5}{r}{{End of table}} \\ 
            \endlastfoot
        }
    },
    %
    end table=\end{longtable},
    col sep=tab,
    string type,
    columns/element_number/.style={
            string type, 
            column type= V{1.475000 in}, 
            string replace*={_}{\_}
        },
    columns/element_name/.style={
            string type, 
            column type= n{1.475000 in}, 
            string replace*={_}{\_}
        },
    columns/type/.style={
            string type, 
            column type= V{1.475000 in}, 
            string replace*={_}{\_}
        },
    columns/external_table/.style={
            string type, 
            column type= n{1.475000 in}, 
            string replace*={_}{\_}
        },
    columns/description/.style={
            string type, 
            string replace*={_}{\_},
            column type = V{3.000000 in}
        }
    ]{/Users/dyb/GitHub/C3S_311a_CDM/tables/sensor_configuration.csv}
\end{landscape}


%\end{landscape}

\section {References}

\noindent WMO, 2015a: Manual On Codes (WMO-No 306), Volume I.2, Part B - Binary Codes, WMO, Geneva.\\
\noindent WMO, 2015b:  Manual on the WMO Integrated Global Observing System: Anenex VIII to the Technical Regulations (WMO-No 1160), WMO, Geneva.


\section {Appendix}

\subsection {Code tables}

\begin{landscape}
\pgfplotstabletypeset[
    empty header,
    begin table=\begin{longtable},
    %every head row/.style={output empty row},
    every nth row={1}{before row=\hline},
    every first row/.append style={
        before row={%
            % Initial caption
            \caption{adjustment}
            \label{tab:DataTableAdjustment}\\
            % Initial column headers
            \hline\hline             \multicolumn{1} { V{0.983333 in}} { \textbf{\seqsplit{index}}} & 
            \multicolumn{1} { V{0.983333 in}} { \textbf{\seqsplit{adjustment}}} & 
            \multicolumn{1} { V{0.983333 in}} { \textbf{\seqsplit{report\_id}}} & 
            \multicolumn{1} { V{0.983333 in}} { \textbf{\seqsplit{observation\_id}}} & 
            \multicolumn{1} { V{0.983333 in}} { \textbf{\seqsplit{value}}} & 
            \multicolumn{1} { V{0.983333 in}} { \textbf{\seqsplit{reason}}} & 
            \multicolumn{1} { V{3.000000 in} } {\textbf{\seqsplit{reference}}} \\ \hline\hline \endfirsthead
            \multicolumn{7}{c}{Table \thetable\ adjustment (cont.)} \\
            % column headers on additional pages
            \hline\hline             \multicolumn{1} {V{0.983333 in} } { \textbf{\seqsplit{index}}} & 
            \multicolumn{1} {V{0.983333 in} } { \textbf{\seqsplit{adjustment}}} & 
            \multicolumn{1} {V{0.983333 in} } { \textbf{\seqsplit{report\_id}}} & 
            \multicolumn{1} {V{0.983333 in} } { \textbf{\seqsplit{observation\_id}}} & 
            \multicolumn{1} {V{0.983333 in} } { \textbf{\seqsplit{value}}} & 
            \multicolumn{1} {V{0.983333 in} } { \textbf{\seqsplit{reason}}} & 
            \multicolumn{1} { V{3.000000 in} } {\textbf{\seqsplit{reference}}} \\ \hline\hline \endhead
            % Footer on 1st to penultimate pages
            \multicolumn{7}{r}{{Continued on next page}} \\
            \endfoot
            % Footer on last page of table
            \hline
            \multicolumn{7}{r}{{End of table}} \\ 
            \endlastfoot
        }
    },
    %
    end table=\end{longtable},
    col sep=tab,
    string type,
    columns/index/.style={
            string type, 
            column type= V{0.983333 in}, 
            string replace*={_}{\_}
        },
    columns/adjustment/.style={
            string type, 
            column type= V{0.983333 in}, 
            string replace*={_}{\_}
        },
    columns/report_id/.style={
            string type, 
            column type= V{0.983333 in}, 
            string replace*={_}{\_}
        },
    columns/observation_id/.style={
            string type, 
            column type= V{0.983333 in}, 
            string replace*={_}{\_}
        },
    columns/value/.style={
            string type, 
            column type= V{0.983333 in}, 
            string replace*={_}{\_}
        },
    columns/reason/.style={
            string type, 
            column type= V{0.983333 in}, 
            string replace*={_}{\_}
        },
    columns/reference/.style={
            string type, 
            string replace*={_}{\_},
            column type = V{3.000000 in}
        }
    ]{/Users/dyb/GitHub/C3S_311a_CDM/tables/adjustment.csv}
\end{landscape}

\pgfplotstabletypeset[
    empty header,
    begin table=\begin{longtable},
    %every head row/.style={output empty row},
    every nth row={1}{before row=\hline},
    every first row/.append style={
        before row={%
            % Initial caption
            \caption{Application area}
            \label{tab:DataTable}\\
            % Initial column headers
            \hline\hline \multicolumn{1} { > {\centering}V{3.300000 in}} { \textbf{Value}} & 
 \multicolumn{1} { > {\centering} V{3.000000 in} } {\textbf{Description}} \\ \hline\hline \endfirsthead
            \multicolumn{2}{c}{Table \thetable\ Application area (cont.)} \\
            % column headers on additional pages
            \hline\hline \multicolumn{1} { > {\centering}V{3.300000 in} } { \textbf{Value}} & 
 \multicolumn{1} { > {\centering} V{3.000000 in} } {\textbf{Description}} \\ \hline\hline \endhead
            % Footer on 1st to penultimate pages
            \multicolumn{2}{r}{{Continued on next page}} \\
            \endfoot
            % Footer on last page of table
            \hline
            \multicolumn{2}{r}{{End of table}} \\ 
            \endlastfoot
        }
    },
    %
    end table=\end{longtable},
    col sep=tab,
    string type,
    columns/Value/.style={
            string type, 
            column type= V{3.300000 in}, 
            string replace*={_}{}
        },
    columns/Description/.style={
            string type, 
            string replace*={_}{},
            column type = V{3.000000 in}
        }
    ]{/Users/dyb/GitHub/C3S_311a_CDM/tables/application_area.csv}

\pgfplotstabletypeset[
    empty header,
    begin table=\begin{longtable},
    %every head row/.style={output empty row},
    every nth row={1}{before row=\hline},
    every first row/.append style={
        before row={%
            % Initial caption
            \caption{Automation status}
            \label{tab:DataTable}\\
            % Initial column headers
            \hline\hline \multicolumn{1} { > {\centering}V{3.300000 in}} { \textbf{Value}} & 
 \multicolumn{1} { > {\centering} V{3.000000 in} } {\textbf{Description}} \\ \hline\hline \endfirsthead
            \multicolumn{2}{c}{Table \thetable\ Automation status (cont.)} \\
            % column headers on additional pages
            \hline\hline \multicolumn{1} { > {\centering}V{3.300000 in} } { \textbf{Value}} & 
 \multicolumn{1} { > {\centering} V{3.000000 in} } {\textbf{Description}} \\ \hline\hline \endhead
            % Footer on 1st to penultimate pages
            \multicolumn{2}{r}{{Continued on next page}} \\
            \endfoot
            % Footer on last page of table
            \hline
            \multicolumn{2}{r}{{End of table}} \\ 
            \endlastfoot
        }
    },
    %
    end table=\end{longtable},
    col sep=tab,
    string type,
    columns/Value/.style={
            string type, 
            column type= V{3.300000 in}, 
            string replace*={_}{}
        },
    columns/Description/.style={
            string type, 
            string replace*={_}{},
            column type = V{3.000000 in}
        }
    ]{/Users/dyb/GitHub/C3S_311a_CDM/tables/automation_status.csv}

\pgfplotstabletypeset[
    empty header,
    begin table=\begin{longtable},
    %every head row/.style={output empty row},
    every nth row={1}{before row=\hline},
    every first row/.append style={
        before row={%
            % Initial caption
            \caption{calibration\_status}
            \label{tab:DataTableCalibrationstatus}\\
            % Initial column headers
            \hline\hline \multicolumn{1} { > {\centering}V{3.300000 in}} { \textbf{Value}} & 
 \multicolumn{1} { > {\centering} V{3.000000 in} } {\textbf{Description}} \\ \hline\hline \endfirsthead
            \multicolumn{2}{c}{Table \thetable\ calibration\_status (cont.)} \\
            % column headers on additional pages
            \hline\hline \multicolumn{1} { > {\centering}V{3.300000 in} } { \textbf{Value}} & 
 \multicolumn{1} { > {\centering} V{3.000000 in} } {\textbf{Description}} \\ \hline\hline \endhead
            % Footer on 1st to penultimate pages
            \multicolumn{2}{r}{{Continued on next page}} \\
            \endfoot
            % Footer on last page of table
            \hline
            \multicolumn{2}{r}{{End of table}} \\ 
            \endlastfoot
        }
    },
    %
    end table=\end{longtable},
    col sep=tab,
    string type,
    columns/Value/.style={
            string type, 
            column type= V{3.300000 in}, 
            string replace*={_}{\_}
        },
    columns/Description/.style={
            string type, 
            string replace*={_}{\_},
            column type = V{3.000000 in}
        }
    ]{/Users/dyb/GitHub/C3S_311a_CDM/tables/calibration_status.csv}

\pgfplotstabletypeset[
    empty header,
    begin table=\begin{longtable},
    %every head row/.style={output empty row},
    every nth row={1}{before row=\hline},
    every first row/.append style={
        before row={%
            % Initial caption
            \caption{Communication method}
            \label{tab:DataTable}\\
            % Initial column headers
            \hline\hline \multicolumn{1} { > {\centering}V{3.300000 in}} { \textbf{Value}} & 
 \multicolumn{1} { > {\centering} V{3.000000 in} } {\textbf{Description}} \\ \hline\hline \endfirsthead
            \multicolumn{2}{c}{Table \thetable\ Communication method (cont.)} \\
            % column headers on additional pages
            \hline\hline \multicolumn{1} { > {\centering}V{3.300000 in} } { \textbf{Value}} & 
 \multicolumn{1} { > {\centering} V{3.000000 in} } {\textbf{Description}} \\ \hline\hline \endhead
            % Footer on 1st to penultimate pages
            \multicolumn{2}{r}{{Continued on next page}} \\
            \endfoot
            % Footer on last page of table
            \hline
            \multicolumn{2}{r}{{End of table}} \\ 
            \endlastfoot
        }
    },
    %
    end table=\end{longtable},
    col sep=tab,
    string type,
    columns/Value/.style={
            string type, 
            column type= V{3.300000 in}, 
            string replace*={_}{}
        },
    columns/Description/.style={
            string type, 
            string replace*={_}{},
            column type = V{3.000000 in}
        }
    ]{/Users/dyb/GitHub/C3S_311a_CDM/tables/communication_method.csv}

\pgfplotstabletypeset[
    empty header,
    begin table=\begin{longtable},
    %every head row/.style={output empty row},
    every nth row={1}{before row=\hline},
    every first row/.append style={
        before row={%
            % Initial caption
            \caption{conversion\_factor}
            \label{tab:DataTableConversionfactor}\\
            % Initial column headers
            \hline\hline \multicolumn{1} { > {\centering}V{1.650000 in}} { \textbf{Value}} & 
\multicolumn{1} { > {\centering}V{1.650000 in}} { \textbf{description}} & 
 \multicolumn{1} { > {\centering} V{3.000000 in} } {\textbf{Implementation}} \\ \hline\hline \endfirsthead
            \multicolumn{3}{c}{Table \thetable\ conversion\_factor (cont.)} \\
            % column headers on additional pages
            \hline\hline \multicolumn{1} { > {\centering}V{1.650000 in} } { \textbf{Value}} & 
\multicolumn{1} { > {\centering}V{1.650000 in} } { \textbf{description}} & 
 \multicolumn{1} { > {\centering} V{3.000000 in} } {\textbf{Implementation}} \\ \hline\hline \endhead
            % Footer on 1st to penultimate pages
            \multicolumn{3}{r}{{Continued on next page}} \\
            \endfoot
            % Footer on last page of table
            \hline
            \multicolumn{3}{r}{{End of table}} \\ 
            \endlastfoot
        }
    },
    %
    end table=\end{longtable},
    col sep=tab,
    string type,
    columns/Value/.style={
            string type, 
            column type= V{1.650000 in}, 
            string replace*={_}{\_}
        },
    columns/description/.style={
            string type, 
            column type= V{1.650000 in}, 
            string replace*={_}{\_}
        },
    columns/Implementation/.style={
            string type, 
            string replace*={_}{\_},
            column type = V{3.000000 in}
        }
    ]{/Users/dyb/GitHub/C3S_311a_CDM/tables/conversion_factor.csv}

\pgfplotstabletypeset[
    empty header,
    begin table=\begin{longtable},
    %every head row/.style={output empty row},
    every nth row={1}{before row=\hline},
    every first row/.append style={
        before row={%
            % Initial caption
            \caption{crs}
            \label{tab:DataTableCrs}\\
            % Initial column headers
            \hline\hline             \multicolumn{1} { V{4.300000 in}} { \textbf{\seqsplit{value}}} & 
            \multicolumn{1} { V{3.000000 in} } {\textbf{\seqsplit{description}}} \\ \hline\hline \endfirsthead
            \multicolumn{2}{c}{Table \thetable\ crs (cont.)} \\
            % column headers on additional pages
            \hline\hline             \multicolumn{1} {V{4.300000 in} } { \textbf{\seqsplit{value}}} & 
            \multicolumn{1} { V{3.000000 in} } {\textbf{\seqsplit{description}}} \\ \hline\hline \endhead
            % Footer on 1st to penultimate pages
            \multicolumn{2}{r}{{Continued on next page}} \\
            \endfoot
            % Footer on last page of table
            \hline
            \multicolumn{2}{r}{{End of table}} \\ 
            \endlastfoot
        }
    },
    %
    end table=\end{longtable},
    col sep=tab,
    string type,
    columns/value/.style={
            string type, 
            column type= V{4.300000 in}, 
            string replace*={_}{\_}
        },
    columns/description/.style={
            string type, 
            string replace*={_}{\_},
            column type = V{3.000000 in}
        }
    ]{/Users/dyb/GitHub/C3S_311a_CDM/tables/crs.csv}

\pgfplotstabletypeset[
    empty header,
    begin table=\begin{longtable},
    %every head row/.style={output empty row},
    every nth row={1}{before row=\hline},
    every first row/.append style={
        before row={%
            % Initial caption
            \caption{data\_policy\_licence}
            \label{tab:DataTableDatapolicylicence}\\
            % Initial column headers
            \hline\hline \multicolumn{1} { > {\centering}V{1.650000 in}} { \textbf{Value}} & 
\multicolumn{1} { > {\centering}V{1.650000 in}} { \textbf{name}} & 
 \multicolumn{1} { > {\centering} V{3.000000 in} } {\textbf{Description}} \\ \hline\hline \endfirsthead
            \multicolumn{3}{c}{Table \thetable\ data\_policy\_licence (cont.)} \\
            % column headers on additional pages
            \hline\hline \multicolumn{1} { > {\centering}V{1.650000 in} } { \textbf{Value}} & 
\multicolumn{1} { > {\centering}V{1.650000 in} } { \textbf{name}} & 
 \multicolumn{1} { > {\centering} V{3.000000 in} } {\textbf{Description}} \\ \hline\hline \endhead
            % Footer on 1st to penultimate pages
            \multicolumn{3}{r}{{Continued on next page}} \\
            \endfoot
            % Footer on last page of table
            \hline
            \multicolumn{3}{r}{{End of table}} \\ 
            \endlastfoot
        }
    },
    %
    end table=\end{longtable},
    col sep=tab,
    string type,
    columns/Value/.style={
            string type, 
            column type= V{1.650000 in}, 
            string replace*={_}{\_}
        },
    columns/name/.style={
            string type, 
            column type= V{1.650000 in}, 
            string replace*={_}{\_}
        },
    columns/Description/.style={
            string type, 
            string replace*={_}{\_},
            column type = V{3.000000 in}
        }
    ]{/Users/dyb/GitHub/C3S_311a_CDM/tables/data_policy_licence.csv}

\pgfplotstabletypeset[
    empty header,
    begin table=\begin{longtable},
    %every head row/.style={output empty row},
    every nth row={1}{before row=\hline},
    every first row/.append style={
        before row={%
            % Initial caption
            \caption{duplicate\_status}
            \label{tab:DataTableDuplicatestatus}\\
            % Initial column headers
            \hline\hline             \multicolumn{1} { V{1.900000 in}} { \textbf{\seqsplit{index}}} & 
            \multicolumn{1} { V{1.900000 in}} { \textbf{\seqsplit{duplicate\_status}}} & 
            \multicolumn{1} { V{3.000000 in} } {\textbf{\seqsplit{description}}} \\ \hline\hline \endfirsthead
            \multicolumn{3}{c}{Table \thetable\ duplicate\_status (cont.)} \\
            % column headers on additional pages
            \hline\hline             \multicolumn{1} {V{1.900000 in} } { \textbf{\seqsplit{index}}} & 
            \multicolumn{1} {V{1.900000 in} } { \textbf{\seqsplit{duplicate\_status}}} & 
            \multicolumn{1} { V{3.000000 in} } {\textbf{\seqsplit{description}}} \\ \hline\hline \endhead
            % Footer on 1st to penultimate pages
            \multicolumn{3}{r}{{Continued on next page}} \\
            \endfoot
            % Footer on last page of table
            \hline
            \multicolumn{3}{r}{{End of table}} \\ 
            \endlastfoot
        }
    },
    %
    end table=\end{longtable},
    col sep=tab,
    string type,
    columns/index/.style={
            string type, 
            column type= V{1.900000 in}, 
            string replace*={_}{\_}
        },
    columns/duplicate_status/.style={
            string type, 
            column type= V{1.900000 in}, 
            string replace*={_}{\_}
        },
    columns/description/.style={
            string type, 
            string replace*={_}{\_},
            column type = V{3.000000 in}
        }
    ]{/Users/dyb/GitHub/C3S_311a_CDM/tables/duplicate_status.csv}


\pgfplotstabletypeset[
    empty header,
    begin table=\begin{longtable},
    %every head row/.style={output empty row},
    every nth row={1}{before row=\hline},
    every first row/.append style={
        before row={%
            % Initial caption
            \caption{events\_at\_station}
            \label{tab:DataTableEventsatstation}\\
            % Initial column headers
            \hline\hline             \multicolumn{1} { V{1.900000 in}} { \textbf{\seqsplit{index}}} & 
            \multicolumn{1} { V{1.900000 in}} { \textbf{\seqsplit{events\_at\_station}}} & 
            \multicolumn{1} { V{3.000000 in} } {\textbf{\seqsplit{description}}} \\ \hline\hline \endfirsthead
            \multicolumn{3}{c}{Table \thetable\ events\_at\_station (cont.)} \\
            % column headers on additional pages
            \hline\hline             \multicolumn{1} {V{1.900000 in} } { \textbf{\seqsplit{index}}} & 
            \multicolumn{1} {V{1.900000 in} } { \textbf{\seqsplit{events\_at\_station}}} & 
            \multicolumn{1} { V{3.000000 in} } {\textbf{\seqsplit{description}}} \\ \hline\hline \endhead
            % Footer on 1st to penultimate pages
            \multicolumn{3}{r}{{Continued on next page}} \\
            \endfoot
            % Footer on last page of table
            \hline
            \multicolumn{3}{r}{{End of table}} \\ 
            \endlastfoot
        }
    },
    %
    end table=\end{longtable},
    col sep=tab,
    string type,
    columns/index/.style={
            string type, 
            column type= V{1.900000 in}, 
            string replace*={_}{\_}
        },
    columns/events_at_station/.style={
            string type, 
            column type= V{1.900000 in}, 
            string replace*={_}{\_}
        },
    columns/description/.style={
            string type, 
            string replace*={_}{\_},
            column type = V{3.000000 in}
        }
    ]{/Users/dyb/GitHub/C3S_311a_CDM/tables/events_at_station.csv}


\documentclass{article}

\usepackage[a4paper, total={6in, 9.5in}]{geometry}
\usepackage{pgfplotstable} 
\usepackage{longtable}
\usepackage{booktabs}
\renewcommand{\familydefault}{\sfdefault}




\pgfplotstableset{
begin table=\begin{longtable},
end table=\end{longtable},
}


\title {Copernicus Climate Change Service - 311a Lot 2\\Defining a common data model}
\author {David I. Berry}
\date {23 June 2017}
\begin{document}
\maketitle
\section {Decode tables}

\begin{table}[h!]
  \begin{center}
    \caption{Id scheme}
    \pgfplotstabletypeset[
        col sep=tab, string type,
        header=true,
        every head row/.style={before row=\toprule\toprule, after row=\bottomrule\bottomrule\endhead}, 
        every last row/.style={after row=\bottomrule},
        after row=\hline,
        columns={Value,Description},
        columns/Value/.style={string type, column type=l, string replace*={_}{}},
        columns/Description/.style={string type, column type=p{0.5\textwidth}, string replace*={_}{}}
    ]{/Users/dyb/Documents/Projects/C3S_311a_Lot2/WP2/CDM/github/tables/id_scheme.csv}
  \end{center}
\end{table}

\end{document}

\begin{landscape}
\pgfplotstabletypeset[
    empty header,
    begin table=\begin{longtable},
    %every head row/.style={output empty row},
    every nth row={1}{before row=\hline},
    every first row/.append style={
        before row={%
            % Initial caption
            \caption{institute}
            \label{tab:DataTableInstitute}\\
            % Initial column headers
            \hline\hline \multicolumn{1} { > {\centering}V{0.957143 in}} { \textbf{Value}} & 
\multicolumn{1} { > {\centering}V{0.957143 in}} { \textbf{Name}} & 
\multicolumn{1} { > {\centering}V{0.957143 in}} { \textbf{Region}} & 
\multicolumn{1} { > {\centering}V{0.957143 in}} { \textbf{Sub region}} & 
\multicolumn{1} { > {\centering}V{0.957143 in}} { \textbf{Address}} & 
\multicolumn{1} { > {\centering}V{0.957143 in}} { \textbf{Contact}} & 
\multicolumn{1} { > {\centering}V{0.957143 in}} { \textbf{Contact email}} & 
 \multicolumn{1} { > {\centering} V{3.000000 in} } {\textbf{URL}} \\ \hline\hline \endfirsthead
            \multicolumn{8}{c}{Table \thetable\ institute (cont.)} \\
            % column headers on additional pages
            \hline\hline \multicolumn{1} { > {\centering}V{0.957143 in} } { \textbf{Value}} & 
\multicolumn{1} { > {\centering}V{0.957143 in} } { \textbf{Name}} & 
\multicolumn{1} { > {\centering}V{0.957143 in} } { \textbf{Region}} & 
\multicolumn{1} { > {\centering}V{0.957143 in} } { \textbf{Sub region}} & 
\multicolumn{1} { > {\centering}V{0.957143 in} } { \textbf{Address}} & 
\multicolumn{1} { > {\centering}V{0.957143 in} } { \textbf{Contact}} & 
\multicolumn{1} { > {\centering}V{0.957143 in} } { \textbf{Contact email}} & 
 \multicolumn{1} { > {\centering} V{3.000000 in} } {\textbf{URL}} \\ \hline\hline \endhead
            % Footer on 1st to penultimate pages
            \multicolumn{8}{r}{{Continued on next page}} \\
            \endfoot
            % Footer on last page of table
            \hline
            \multicolumn{8}{r}{{End of table}} \\ 
            \endlastfoot
        }
    },
    %
    end table=\end{longtable},
    col sep=tab,
    string type,
    columns/Value/.style={
            string type, 
            column type= V{0.957143 in}, 
            string replace*={_}{\_}
        },
    columns/Name/.style={
            string type, 
            column type= V{0.957143 in}, 
            string replace*={_}{\_}
        },
    columns/Region/.style={
            string type, 
            column type= V{0.957143 in}, 
            string replace*={_}{\_}
        },
    columns/Sub region/.style={
            string type, 
            column type= V{0.957143 in}, 
            string replace*={_}{\_}
        },
    columns/Address/.style={
            string type, 
            column type= V{0.957143 in}, 
            string replace*={_}{\_}
        },
    columns/Contact/.style={
            string type, 
            column type= V{0.957143 in}, 
            string replace*={_}{\_}
        },
    columns/Contact email/.style={
            string type, 
            column type= V{0.957143 in}, 
            string replace*={_}{\_}
        },
    columns/URL/.style={
            string type, 
            string replace*={_}{\_},
            column type = V{3.000000 in}
        }
    ]{/Users/dyb/GitHub/C3S_311a_CDM/tables/institute.csv}
\end{landscape}

\pgfplotstabletypeset[
    empty header,
    begin table=\begin{longtable},
    %every head row/.style={output empty row},
    every nth row={1}{before row=\hline},
    every first row/.append style={
        before row={%
            % Initial caption
            \caption{instrument\_exposure\_quality}
            \label{tab:DataTableInstrumentexposurequality}\\
            % Initial column headers
            \hline\hline \multicolumn{1} { > {\centering}V{3.300000 in}} { \textbf{value}} & 
 \multicolumn{1} { > {\centering} V{3.000000 in} } {\textbf{description}} \\ \hline\hline \endfirsthead
            \multicolumn{2}{c}{Table \thetable\ instrument\_exposure\_quality (cont.)} \\
            % column headers on additional pages
            \hline\hline \multicolumn{1} { > {\centering}V{3.300000 in} } { \textbf{value}} & 
 \multicolumn{1} { > {\centering} V{3.000000 in} } {\textbf{description}} \\ \hline\hline \endhead
            % Footer on 1st to penultimate pages
            \multicolumn{2}{r}{{Continued on next page}} \\
            \endfoot
            % Footer on last page of table
            \hline
            \multicolumn{2}{r}{{End of table}} \\ 
            \endlastfoot
        }
    },
    %
    end table=\end{longtable},
    col sep=tab,
    string type,
    columns/value/.style={
            string type, 
            column type= V{3.300000 in}, 
            string replace*={_}{\_}
        },
    columns/description/.style={
            string type, 
            string replace*={_}{\_},
            column type = V{3.000000 in}
        }
    ]{/Users/dyb/GitHub/C3S_311a_CDM/tables/instrument_exposure_quality.csv}

\pgfplotstabletypeset[
    empty header,
    begin table=\begin{longtable},
    %every head row/.style={output empty row},
    every nth row={1}{before row=\hline},
    every first row/.append style={
        before row={%
            % Initial caption
            \caption{Location method}
            \label{tab:DataTable}\\
            % Initial column headers
            \hline\hline \multicolumn{1} { > {\centering}V{3.300000 in}} { \textbf{Value}} & 
 \multicolumn{1} { > {\centering} V{3.000000 in} } {\textbf{Description}} \\ \hline\hline \endfirsthead
            \multicolumn{2}{c}{Table \thetable\ Location method (cont.)} \\
            % column headers on additional pages
            \hline\hline \multicolumn{1} { > {\centering}V{3.300000 in} } { \textbf{Value}} & 
 \multicolumn{1} { > {\centering} V{3.000000 in} } {\textbf{Description}} \\ \hline\hline \endhead
            % Footer on 1st to penultimate pages
            \multicolumn{2}{r}{{Continued on next page}} \\
            \endfoot
            % Footer on last page of table
            \hline
            \multicolumn{2}{r}{{End of table}} \\ 
            \endlastfoot
        }
    },
    %
    end table=\end{longtable},
    col sep=tab,
    string type,
    columns/Value/.style={
            string type, 
            column type= V{3.300000 in}, 
            string replace*={_}{}
        },
    columns/Description/.style={
            string type, 
            string replace*={_}{},
            column type = V{3.000000 in}
        }
    ]{/Users/dyb/GitHub/C3S_311a_CDM/tables/location_method.csv}

\pgfplotstabletypeset[
    empty header,
    begin table=\begin{longtable},
    %every head row/.style={output empty row},
    every nth row={1}{before row=\hline},
    every first row/.append style={
        before row={%
            % Initial caption
            \caption{Location quality}
            \label{tab:DataTable}\\
            % Initial column headers
            \hline\hline \multicolumn{1} { > {\centering}V{3.300000 in}} { \textbf{Value}} & 
 \multicolumn{1} { > {\centering} V{3.000000 in} } {\textbf{Description}} \\ \hline\hline \endfirsthead
            \multicolumn{2}{c}{Table \thetable\ Location quality (cont.)} \\
            % column headers on additional pages
            \hline\hline \multicolumn{1} { > {\centering}V{3.300000 in} } { \textbf{Value}} & 
 \multicolumn{1} { > {\centering} V{3.000000 in} } {\textbf{Description}} \\ \hline\hline \endhead
            % Footer on 1st to penultimate pages
            \multicolumn{2}{r}{{Continued on next page}} \\
            \endfoot
            % Footer on last page of table
            \hline
            \multicolumn{2}{r}{{End of table}} \\ 
            \endlastfoot
        }
    },
    %
    end table=\end{longtable},
    col sep=tab,
    string type,
    columns/Value/.style={
            string type, 
            column type= V{3.300000 in}, 
            string replace*={_}{}
        },
    columns/Description/.style={
            string type, 
            string replace*={_}{},
            column type = V{3.000000 in}
        }
    ]{/Users/dyb/GitHub/C3S_311a_CDM/tables/location_quality.csv}

% \pgfplotstabletypeset[
    empty header,
    begin table=\begin{longtable},
    %every head row/.style={output empty row},
    every nth row={1}{before row=\hline},
    every first row/.append style={
        before row={%
            % Initial caption
            \caption{metadata\_source}
            \label{tab:DataTableMetadatasource}\\
            % Initial column headers
            \hline\hline             \multicolumn{1} { V{1.433333 in}} { \textbf{\seqsplit{value}}} & 
            \multicolumn{1} { V{1.433333 in}} { \textbf{\seqsplit{description}}} & 
            \multicolumn{1} { V{1.433333 in}} { \textbf{\seqsplit{version}}} & 
            \multicolumn{1} { V{3.000000 in} } {\textbf{\seqsplit{URL}}} \\ \hline\hline \endfirsthead
            \multicolumn{4}{c}{Table \thetable\ metadata\_source (cont.)} \\
            % column headers on additional pages
            \hline\hline             \multicolumn{1} {V{1.433333 in} } { \textbf{\seqsplit{value}}} & 
            \multicolumn{1} {V{1.433333 in} } { \textbf{\seqsplit{description}}} & 
            \multicolumn{1} {V{1.433333 in} } { \textbf{\seqsplit{version}}} & 
            \multicolumn{1} { V{3.000000 in} } {\textbf{\seqsplit{URL}}} \\ \hline\hline \endhead
            % Footer on 1st to penultimate pages
            \multicolumn{4}{r}{{Continued on next page}} \\
            \endfoot
            % Footer on last page of table
            \hline
            \multicolumn{4}{r}{{End of table}} \\ 
            \endlastfoot
        }
    },
    %
    end table=\end{longtable},
    col sep=tab,
    string type,
    columns/value/.style={
            string type, 
            column type= V{1.433333 in}, 
            string replace*={_}{\_}
        },
    columns/description/.style={
            string type, 
            column type= V{1.433333 in}, 
            string replace*={_}{\_}
        },
    columns/version/.style={
            string type, 
            column type= V{1.433333 in}, 
            string replace*={_}{\_}
        },
    columns/URL/.style={
            string type, 
            string replace*={_}{\_},
            column type = V{3.000000 in}
        }
    ]{/Users/dyb/GitHub/C3S_311a_CDM/tables/metadata_source.csv}

\pgfplotstabletypeset[
    empty header,
    begin table=\begin{longtable},
    every head row/.style={output empty row},
    every nth row={1}{before row=\hline},
    every first row/.append style={
        before row={%
            % Initial caption
            \caption{Meaning of time stamp}
            \label{tab:DataTable}\\
            % Initial column headers
            \hline\hline \multicolumn{1} { > {\centering}V{2 in}} { \textbf{Value}} &\multicolumn{1} { > {\centering}V{2 in}} { \textbf{name}} & \multicolumn{1} { > {\centering} V{4 in} } {\textbf{Description}} \\ \hline\hline \endfirsthead
            \multicolumn{3}{c}{Table \thetable\ Meaning of time stamp (cont.)} \\
            % column headers on additional pages
            \hline\hline \multicolumn{1} { > {\centering}V{2 in} } { \textbf{Value}} &\multicolumn{1} { > {\centering}V{2 in} } { \textbf{name}} & \multicolumn{1} { > {\centering} V{4 in} } {\textbf{Description}} \\ \hline\hline \endhead
            % Footer on 1st to penultimate pages
            \multicolumn{3}{r}{{Continued on next page}} \\
            \endfoot
            % Footer on last page of table
            \hline
            \multicolumn{3}{r}{{End of table}} \\ 
            \endlastfoot
        }
    },
    %
    end table=\end{longtable},
    col sep=tab,
    string type,
    columns/Value/.style={
            string type, 
            column type= V{2 in}, 
            string replace*={_}{}
        },
    columns/name/.style={
            string type, 
            column type= V{2 in}, 
            string replace*={_}{}
        },
    columns/Description/.style={
            string type, 
            string replace*={_}{},
            column type = V{4 in}
        }
    ]{/Users/dyb/GitHub/C3S_311a_CDM/tables/meaning_of_time_stamp.csv}

\documentclass{article}

\usepackage[a4paper, total={6in, 9.5in}]{geometry}
\usepackage{pgfplotstable} 
\usepackage{longtable}
\usepackage{booktabs}
\renewcommand{\familydefault}{\sfdefault}




\pgfplotstableset{
begin table=\begin{longtable},
end table=\end{longtable},
}


\title {Copernicus Climate Change Service - 311a Lot 2\\Defining a common data model}
\author {David I. Berry}
\date {23 June 2017}
\begin{document}
\maketitle
\section {Decode tables}

\begin{table}[h!]
  \begin{center}
    \caption{Observed variable}
    \pgfplotstabletypeset[
        col sep=tab, string type,
        header=true,
        every head row/.style={before row=\toprule\toprule, after row=\bottomrule\bottomrule\endhead}, 
        every last row/.style={after row=\bottomrule},
        after row=\hline,
        columns={Value,Abbreviation,Standard name,Units,Description,Code space,Code space index},
        columns/Value/.style={string type, column type=l, string replace*={_}{}},
        columns/Abbreviation/.style={string type, column type=l, string replace*={_}{}},
        columns/Standard name/.style={string type, column type=l, string replace*={_}{}},
        columns/Units/.style={string type, column type=l, string replace*={_}{}},
        columns/Description/.style={string type, column type=l, string replace*={_}{}},
        columns/Code space/.style={string type, column type=l, string replace*={_}{}},
        columns/Code space index/.style={string type, column type=p{0.5\textwidth}, string replace*={_}{}}
    ]{/Users/dyb/Documents/Projects/C3S_311a_Lot2/WP2/CDM/github/tables/observed_variable.csv}
  \end{center}
\end{table}

\end{document}

\pgfplotstabletypeset[
    empty header,
    begin table=\begin{longtable},
    %every head row/.style={output empty row},
    every nth row={1}{before row=\hline},
    every first row/.append style={
        before row={%
            % Initial caption
            \caption{Observation value significance}
            \label{tab:DataTable}\\
            % Initial column headers
            \hline\hline \multicolumn{1} { > {\centering}V{3.300000 in}} { \textbf{Value}} & 
 \multicolumn{1} { > {\centering} V{3.000000 in} } {\textbf{Description}} \\ \hline\hline \endfirsthead
            \multicolumn{2}{c}{Table \thetable\ Observation value significance (cont.)} \\
            % column headers on additional pages
            \hline\hline \multicolumn{1} { > {\centering}V{3.300000 in} } { \textbf{Value}} & 
 \multicolumn{1} { > {\centering} V{3.000000 in} } {\textbf{Description}} \\ \hline\hline \endhead
            % Footer on 1st to penultimate pages
            \multicolumn{2}{r}{{Continued on next page}} \\
            \endfoot
            % Footer on last page of table
            \hline
            \multicolumn{2}{r}{{End of table}} \\ 
            \endlastfoot
        }
    },
    %
    end table=\end{longtable},
    col sep=tab,
    string type,
    columns/Value/.style={
            string type, 
            column type= V{3.300000 in}, 
            string replace*={_}{}
        },
    columns/Description/.style={
            string type, 
            string replace*={_}{},
            column type = V{3.000000 in}
        }
    ]{/Users/dyb/Documents/Projects/C3S_311a_Lot2/WP2/CDM/github/tables/observation_value_significance.csv}

\documentclass{article}

\usepackage[a4paper, total={6in, 9.5in}]{geometry}
\usepackage{pgfplotstable} 
\usepackage{longtable}
\usepackage{booktabs}
\renewcommand{\familydefault}{\sfdefault}




\pgfplotstableset{
begin table=\begin{longtable},
end table=\end{longtable},
}


\title {Copernicus Climate Change Service - 311a Lot 2\\Defining a common data model}
\author {David I. Berry}
\date {23 June 2017}
\begin{document}
\maketitle
\section {Decode tables}

\begin{table}[h!]
  \begin{center}
    \caption{Observing frequency}
    \pgfplotstabletypeset[
        col sep=tab, string type,
        header=true,
        every head row/.style={before row=\toprule\toprule, after row=\bottomrule\bottomrule\endhead}, 
        every last row/.style={after row=\bottomrule},
        after row=\hline,
        columns={Value,Code,Description},
        columns/Value/.style={string type, column type=l, string replace*={_}{}},
        columns/Code/.style={string type, column type=l, string replace*={_}{}},
        columns/Description/.style={string type, column type=p{0.5\textwidth}, string replace*={_}{}}
    ]{/Users/dyb/Documents/Projects/C3S_311a_Lot2/WP2/CDM/github/tables/observing_frequency.csv}
  \end{center}
\end{table}

\end{document}

\pgfplotstabletypeset[
    empty header,
    begin table=\begin{longtable},
    %every head row/.style={output empty row},
    every nth row={1}{before row=\hline},
    every first row/.append style={
        before row={%
            % Initial caption
            \caption{observing\_method}
            \label{tab:DataTableObservingmethod}\\
            % Initial column headers
            \hline\hline             \multicolumn{1} { V{1.900000 in}} { \textbf{\seqsplit{index}}} & 
            \multicolumn{1} { V{1.900000 in}} { \textbf{\seqsplit{observing\_method}}} & 
            \multicolumn{1} { V{3.000000 in} } {\textbf{\seqsplit{description}}} \\ \hline\hline \endfirsthead
            \multicolumn{3}{c}{Table \thetable\ observing\_method (cont.)} \\
            % column headers on additional pages
            \hline\hline             \multicolumn{1} {V{1.900000 in} } { \textbf{\seqsplit{index}}} & 
            \multicolumn{1} {V{1.900000 in} } { \textbf{\seqsplit{observing\_method}}} & 
            \multicolumn{1} { V{3.000000 in} } {\textbf{\seqsplit{description}}} \\ \hline\hline \endhead
            % Footer on 1st to penultimate pages
            \multicolumn{3}{r}{{Continued on next page}} \\
            \endfoot
            % Footer on last page of table
            \hline
            \multicolumn{3}{r}{{End of table}} \\ 
            \endlastfoot
        }
    },
    %
    end table=\end{longtable},
    col sep=tab,
    string type,
    columns/index/.style={
            string type, 
            column type= V{1.900000 in}, 
            string replace*={_}{\_}
        },
    columns/observing_method/.style={
            string type, 
            column type= V{1.900000 in}, 
            string replace*={_}{\_}
        },
    columns/description/.style={
            string type, 
            string replace*={_}{\_},
            column type = V{3.000000 in}
        }
    ]{/Users/dyb/GitHub/C3S_311a_CDM/tables/observing_method.csv}

\pgfplotstabletypeset[
    empty header,
    begin table=\begin{longtable},
    %every head row/.style={output empty row},
    every nth row={1}{before row=\hline},
    every first row/.append style={
        before row={%
            % Initial caption
            \caption{observing\_programme}
            \label{tab:DataTableObservingprogramme}\\
            % Initial column headers
            \hline\hline \multicolumn{1} { > {\centering}V{1.100000 in}} { \textbf{value}} & 
\multicolumn{1} { > {\centering}V{1.100000 in}} { \textbf{abbreviation}} & 
\multicolumn{1} { > {\centering}V{1.100000 in}} { \textbf{description}} & 
 \multicolumn{1} { > {\centering} V{3.000000 in} } {\textbf{sponsor}} \\ \hline\hline \endfirsthead
            \multicolumn{4}{c}{Table \thetable\ observing\_programme (cont.)} \\
            % column headers on additional pages
            \hline\hline \multicolumn{1} { > {\centering}V{1.100000 in} } { \textbf{value}} & 
\multicolumn{1} { > {\centering}V{1.100000 in} } { \textbf{abbreviation}} & 
\multicolumn{1} { > {\centering}V{1.100000 in} } { \textbf{description}} & 
 \multicolumn{1} { > {\centering} V{3.000000 in} } {\textbf{sponsor}} \\ \hline\hline \endhead
            % Footer on 1st to penultimate pages
            \multicolumn{4}{r}{{Continued on next page}} \\
            \endfoot
            % Footer on last page of table
            \hline
            \multicolumn{4}{r}{{End of table}} \\ 
            \endlastfoot
        }
    },
    %
    end table=\end{longtable},
    col sep=tab,
    string type,
    columns/value/.style={
            string type, 
            column type= V{1.100000 in}, 
            string replace*={_}{\_}
        },
    columns/abbreviation/.style={
            string type, 
            column type= V{1.100000 in}, 
            string replace*={_}{\_}
        },
    columns/description/.style={
            string type, 
            column type= V{1.100000 in}, 
            string replace*={_}{\_}
        },
    columns/sponsor/.style={
            string type, 
            string replace*={_}{\_},
            column type = V{3.000000 in}
        }
    ]{/Users/dyb/GitHub/C3S_311a_CDM/tables/observing_programme.csv}

\pgfplotstabletypeset[
    empty header,
    begin table=\begin{longtable},
    %every head row/.style={output empty row},
    every nth row={1}{before row=\hline},
    every first row/.append style={
        before row={%
            % Initial caption
            \caption{Platform sub type}
            \label{tab:DataTable}\\
            % Initial column headers
            \hline\hline \multicolumn{1} { > {\centering}V{1.100000 in}} { \textbf{Value}} & 
\multicolumn{1} { > {\centering}V{1.100000 in}} { \textbf{Platform Type}} & 
\multicolumn{1} { > {\centering}V{1.100000 in}} { \textbf{Abbreviation}} & 
 \multicolumn{1} { > {\centering} V{3.000000 in} } {\textbf{Description}} \\ \hline\hline \endfirsthead
            \multicolumn{4}{c}{Table \thetable\ Platform sub type (cont.)} \\
            % column headers on additional pages
            \hline\hline \multicolumn{1} { > {\centering}V{1.100000 in} } { \textbf{Value}} & 
\multicolumn{1} { > {\centering}V{1.100000 in} } { \textbf{Platform Type}} & 
\multicolumn{1} { > {\centering}V{1.100000 in} } { \textbf{Abbreviation}} & 
 \multicolumn{1} { > {\centering} V{3.000000 in} } {\textbf{Description}} \\ \hline\hline \endhead
            % Footer on 1st to penultimate pages
            \multicolumn{4}{r}{{Continued on next page}} \\
            \endfoot
            % Footer on last page of table
            \hline
            \multicolumn{4}{r}{{End of table}} \\ 
            \endlastfoot
        }
    },
    %
    end table=\end{longtable},
    col sep=tab,
    string type,
    columns/Value/.style={
            string type, 
            column type= V{1.100000 in}, 
            string replace*={_}{}
        },
    columns/Platform Type/.style={
            string type, 
            column type= V{1.100000 in}, 
            string replace*={_}{}
        },
    columns/Abbreviation/.style={
            string type, 
            column type= V{1.100000 in}, 
            string replace*={_}{}
        },
    columns/Description/.style={
            string type, 
            string replace*={_}{},
            column type = V{3.000000 in}
        }
    ]{/Users/dyb/Documents/Projects/C3S_311a_Lot2/WP2/CDM/github/tables/platform_sub_type.csv}

\pgfplotstabletypeset[
    empty header,
    begin table=\begin{longtable},
    %every head row/.style={output empty row},
    every nth row={1}{before row=\hline},
    every first row/.append style={
        before row={%
            % Initial caption
            \caption{platform\_type}
            \label{tab:DataTablePlatformtype}\\
            % Initial column headers
            \hline\hline \multicolumn{1} { > {\centering}V{3.300000 in}} { \textbf{value}} & 
 \multicolumn{1} { > {\centering} V{3.000000 in} } {\textbf{description}} \\ \hline\hline \endfirsthead
            \multicolumn{2}{c}{Table \thetable\ platform\_type (cont.)} \\
            % column headers on additional pages
            \hline\hline \multicolumn{1} { > {\centering}V{3.300000 in} } { \textbf{value}} & 
 \multicolumn{1} { > {\centering} V{3.000000 in} } {\textbf{description}} \\ \hline\hline \endhead
            % Footer on 1st to penultimate pages
            \multicolumn{2}{r}{{Continued on next page}} \\
            \endfoot
            % Footer on last page of table
            \hline
            \multicolumn{2}{r}{{End of table}} \\ 
            \endlastfoot
        }
    },
    %
    end table=\end{longtable},
    col sep=tab,
    string type,
    columns/value/.style={
            string type, 
            column type= V{3.300000 in}, 
            string replace*={_}{\_}
        },
    columns/description/.style={
            string type, 
            string replace*={_}{\_},
            column type = V{3.000000 in}
        }
    ]{/Users/dyb/GitHub/C3S_311a_CDM/tables/platform_type.csv}

\pgfplotstabletypeset[
    empty header,
    begin table=\begin{longtable},
    %every head row/.style={output empty row},
    every nth row={1}{before row=\hline},
    every first row/.append style={
        before row={%
            % Initial caption
            \caption{processing\_level}
            \label{tab:DataTableProcessinglevel}\\
            % Initial column headers
            \hline\hline             \multicolumn{1} { V{1.900000 in}} { \textbf{\seqsplit{value}}} & 
            \multicolumn{1} { V{1.900000 in}} { \textbf{\seqsplit{processing\_level}}} & 
            \multicolumn{1} { V{3.000000 in} } {\textbf{\seqsplit{description}}} \\ \hline\hline \endfirsthead
            \multicolumn{3}{c}{Table \thetable\ processing\_level (cont.)} \\
            % column headers on additional pages
            \hline\hline             \multicolumn{1} {V{1.900000 in} } { \textbf{\seqsplit{value}}} & 
            \multicolumn{1} {V{1.900000 in} } { \textbf{\seqsplit{processing\_level}}} & 
            \multicolumn{1} { V{3.000000 in} } {\textbf{\seqsplit{description}}} \\ \hline\hline \endhead
            % Footer on 1st to penultimate pages
            \multicolumn{3}{r}{{Continued on next page}} \\
            \endfoot
            % Footer on last page of table
            \hline
            \multicolumn{3}{r}{{End of table}} \\ 
            \endlastfoot
        }
    },
    %
    end table=\end{longtable},
    col sep=tab,
    string type,
    columns/value/.style={
            string type, 
            column type= V{1.900000 in}, 
            string replace*={_}{\_}
        },
    columns/processing_level/.style={
            string type, 
            column type= V{1.900000 in}, 
            string replace*={_}{\_}
        },
    columns/description/.style={
            string type, 
            string replace*={_}{\_},
            column type = V{3.000000 in}
        }
    ]{/Users/dyb/GitHub/C3S_311a_CDM/tables/processing_level.csv}

\begin{landscape}
\pgfplotstabletypeset[
    empty header,
    begin table=\begin{longtable},
    %every head row/.style={output empty row},
    every nth row={1}{before row=\hline},
    every first row/.append style={
        before row={%
            % Initial caption
            \caption{profile\_configuration\_fields}
            \label{tab:DataTableProfileconfigurationfields}\\
            % Initial column headers
            \hline\hline             \multicolumn{1} { V{1.475000 in}} { \textbf{\seqsplit{index}}} & 
            \multicolumn{1} { V{1.475000 in}} { \textbf{\seqsplit{field}}} & 
            \multicolumn{1} { V{1.475000 in}} { \textbf{\seqsplit{field\_name}}} & 
            \multicolumn{1} { V{1.475000 in}} { \textbf{\seqsplit{type}}} & 
            \multicolumn{1} { V{3.000000 in} } {\textbf{\seqsplit{description}}} \\ \hline\hline \endfirsthead
            \multicolumn{5}{c}{Table \thetable\ profile\_configuration\_fields (cont.)} \\
            % column headers on additional pages
            \hline\hline             \multicolumn{1} {V{1.475000 in} } { \textbf{\seqsplit{index}}} & 
            \multicolumn{1} {V{1.475000 in} } { \textbf{\seqsplit{field}}} & 
            \multicolumn{1} {V{1.475000 in} } { \textbf{\seqsplit{field\_name}}} & 
            \multicolumn{1} {V{1.475000 in} } { \textbf{\seqsplit{type}}} & 
            \multicolumn{1} { V{3.000000 in} } {\textbf{\seqsplit{description}}} \\ \hline\hline \endhead
            % Footer on 1st to penultimate pages
            \multicolumn{5}{r}{{Continued on next page}} \\
            \endfoot
            % Footer on last page of table
            \hline
            \multicolumn{5}{r}{{End of table}} \\ 
            \endlastfoot
        }
    },
    %
    end table=\end{longtable},
    col sep=tab,
    string type,
    columns/index/.style={
            string type, 
            column type= V{1.475000 in}, 
            string replace*={_}{\_}
        },
    columns/field/.style={
            string type, 
            column type= V{1.475000 in}, 
            string replace*={_}{\_}
        },
    columns/field_name/.style={
            string type, 
            column type= n{1.475000 in}, 
            string replace*={_}{\_}
        },
    columns/type/.style={
            string type, 
            column type= V{1.475000 in}, 
            string replace*={_}{\_}
        },
    columns/description/.style={
            string type, 
            string replace*={_}{\_},
            column type = V{3.000000 in}
        }
    ]{/Users/dyb/GitHub/C3S_311a_CDM/tables/profile_configuration_fields.csv}
\end{landscape}

\pgfplotstabletypeset[
    empty header,
    begin table=\begin{longtable},
    every head row/.style={output empty row},
    every nth row={1}{before row=\hline},
    every first row/.append style={
        before row={%
            % Initial caption
            \caption{Quality flag}
            \label{tab:DataTable}\\
            % Initial column headers
            \hline\hline \multicolumn{1} { > {\centering}V{2 in}} { \textbf{Value}} & \multicolumn{1} { > {\centering} V{4 in} } {\textbf{Description}} \\ \hline\hline \endfirsthead
            \multicolumn{2}{c}{Table \thetable\ Quality flag (cont.)} \\
            % column headers on additional pages
            \hline\hline \multicolumn{1} { > {\centering}V{2 in} } { \textbf{Value}} & \multicolumn{1} { > {\centering} V{4 in} } {\textbf{Description}} \\ \hline\hline \endhead
            % Footer on 1st to penultimate pages
            \multicolumn{2}{r}{{Continued on next page}} \\
            \endfoot
            % Footer on last page of table
            \hline
            \multicolumn{2}{r}{{End of table}} \\ 
            \endlastfoot
        }
    },
    %
    end table=\end{longtable},
    col sep=tab,
    string type,
    columns/Value/.style={
            string type, 
            column type= V{2 in}, 
            string replace*={_}{}
        },
    columns/Description/.style={
            string type, 
            string replace*={_}{},
            column type = V{4 in}
        }
    ]{/Users/dyb/GitHub/C3S_311a_CDM/tables/quality_flag.csv}

\pgfplotstabletypeset[
    empty header,
    begin table=\begin{longtable},
    %every head row/.style={output empty row},
    every nth row={1}{before row=\hline},
    every first row/.append style={
        before row={%
            % Initial caption
            \caption{region}
            \label{tab:DataTableRegion}\\
            % Initial column headers
            \hline\hline             \multicolumn{1} { V{1.900000 in}} { \textbf{\seqsplit{value}}} & 
            \multicolumn{1} { V{1.900000 in}} { \textbf{\seqsplit{WMO\_region}}} & 
            \multicolumn{1} { V{3.000000 in} } {\textbf{\seqsplit{description}}} \\ \hline\hline \endfirsthead
            \multicolumn{3}{c}{Table \thetable\ region (cont.)} \\
            % column headers on additional pages
            \hline\hline             \multicolumn{1} {V{1.900000 in} } { \textbf{\seqsplit{value}}} & 
            \multicolumn{1} {V{1.900000 in} } { \textbf{\seqsplit{WMO\_region}}} & 
            \multicolumn{1} { V{3.000000 in} } {\textbf{\seqsplit{description}}} \\ \hline\hline \endhead
            % Footer on 1st to penultimate pages
            \multicolumn{3}{r}{{Continued on next page}} \\
            \endfoot
            % Footer on last page of table
            \hline
            \multicolumn{3}{r}{{End of table}} \\ 
            \endlastfoot
        }
    },
    %
    end table=\end{longtable},
    col sep=tab,
    string type,
    columns/value/.style={
            string type, 
            column type= V{1.900000 in}, 
            string replace*={_}{\_}
        },
    columns/WMO_region/.style={
            string type, 
            column type= V{1.900000 in}, 
            string replace*={_}{\_}
        },
    columns/description/.style={
            string type, 
            string replace*={_}{\_},
            column type = V{3.000000 in}
        }
    ]{/Users/dyb/GitHub/C3S_311a_CDM/tables/region.csv}

\pgfplotstabletypeset[
    empty header,
    begin table=\begin{longtable},
    %every head row/.style={output empty row},
    every nth row={1}{before row=\hline},
    every first row/.append style={
        before row={%
            % Initial caption
            \caption{report\_processing\_codes}
            \label{tab:DataTableReportprocessingcodes}\\
            % Initial column headers
            \hline\hline             \multicolumn{1} { V{1.900000 in}} { \textbf{\seqsplit{index}}} & 
            \multicolumn{1} { V{1.900000 in}} { \textbf{\seqsplit{report\_processing\_codes}}} & 
            \multicolumn{1} { V{3.000000 in} } {\textbf{\seqsplit{description}}} \\ \hline\hline \endfirsthead
            \multicolumn{3}{c}{Table \thetable\ report\_processing\_codes (cont.)} \\
            % column headers on additional pages
            \hline\hline             \multicolumn{1} {V{1.900000 in} } { \textbf{\seqsplit{index}}} & 
            \multicolumn{1} {V{1.900000 in} } { \textbf{\seqsplit{report\_processing\_codes}}} & 
            \multicolumn{1} { V{3.000000 in} } {\textbf{\seqsplit{description}}} \\ \hline\hline \endhead
            % Footer on 1st to penultimate pages
            \multicolumn{3}{r}{{Continued on next page}} \\
            \endfoot
            % Footer on last page of table
            \hline
            \multicolumn{3}{r}{{End of table}} \\ 
            \endlastfoot
        }
    },
    %
    end table=\end{longtable},
    col sep=tab,
    string type,
    columns/index/.style={
            string type, 
            column type= V{1.900000 in}, 
            string replace*={_}{\_}
        },
    columns/report_processing_codes/.style={
            string type, 
            column type= V{1.900000 in}, 
            string replace*={_}{\_}
        },
    columns/description/.style={
            string type, 
            string replace*={_}{\_},
            column type = V{3.000000 in}
        }
    ]{/Users/dyb/GitHub/C3S_311a_CDM/tables/report_processing_codes.csv}

\pgfplotstabletypeset[
    empty header,
    begin table=\begin{longtable},
    %every head row/.style={output empty row},
    every nth row={1}{before row=\hline},
    every first row/.append style={
        before row={%
            % Initial caption
            \caption{report\_processing\_level}
            \label{tab:DataTableReportprocessinglevel}\\
            % Initial column headers
            \hline\hline \multicolumn{1} { > {\centering}V{3.300000 in}} { \textbf{value}} & 
 \multicolumn{1} { > {\centering} V{3.000000 in} } {\textbf{description}} \\ \hline\hline \endfirsthead
            \multicolumn{2}{c}{Table \thetable\ report\_processing\_level (cont.)} \\
            % column headers on additional pages
            \hline\hline \multicolumn{1} { > {\centering}V{3.300000 in} } { \textbf{value}} & 
 \multicolumn{1} { > {\centering} V{3.000000 in} } {\textbf{description}} \\ \hline\hline \endhead
            % Footer on 1st to penultimate pages
            \multicolumn{2}{r}{{Continued on next page}} \\
            \endfoot
            % Footer on last page of table
            \hline
            \multicolumn{2}{r}{{End of table}} \\ 
            \endlastfoot
        }
    },
    %
    end table=\end{longtable},
    col sep=tab,
    string type,
    columns/value/.style={
            string type, 
            column type= V{3.300000 in}, 
            string replace*={_}{\_}
        },
    columns/description/.style={
            string type, 
            string replace*={_}{\_},
            column type = V{3.000000 in}
        }
    ]{/Users/dyb/GitHub/C3S_311a_CDM/tables/report_processing_level.csv}

\pgfplotstabletypeset[
    empty header,
    begin table=\begin{longtable},
    %every head row/.style={output empty row},
    every nth row={1}{before row=\hline},
    every first row/.append style={
        before row={%
            % Initial caption
            \caption{report\_type}
            \label{tab:DataTableReporttype}\\
            % Initial column headers
            \hline\hline \multicolumn{1} { > {\centering}V{3.300000 in}} { \textbf{Value}} & 
 \multicolumn{1} { > {\centering} V{3.000000 in} } {\textbf{Description}} \\ \hline\hline \endfirsthead
            \multicolumn{2}{c}{Table \thetable\ report\_type (cont.)} \\
            % column headers on additional pages
            \hline\hline \multicolumn{1} { > {\centering}V{3.300000 in} } { \textbf{Value}} & 
 \multicolumn{1} { > {\centering} V{3.000000 in} } {\textbf{Description}} \\ \hline\hline \endhead
            % Footer on 1st to penultimate pages
            \multicolumn{2}{r}{{Continued on next page}} \\
            \endfoot
            % Footer on last page of table
            \hline
            \multicolumn{2}{r}{{End of table}} \\ 
            \endlastfoot
        }
    },
    %
    end table=\end{longtable},
    col sep=tab,
    string type,
    columns/Value/.style={
            string type, 
            column type= V{3.300000 in}, 
            string replace*={_}{\_}
        },
    columns/Description/.style={
            string type, 
            string replace*={_}{\_},
            column type = V{3.000000 in}
        }
    ]{/Users/dyb/GitHub/C3S_311a_CDM/tables/report_type.csv}

\pgfplotstabletypeset[
    empty header,
    begin table=\begin{longtable},
    %every head row/.style={output empty row},
    every nth row={1}{before row=\hline},
    every first row/.append style={
        before row={%
            % Initial caption
            \caption{sampling\_strategy}
            \label{tab:DataTableSamplingstrategy}\\
            % Initial column headers
            \hline\hline \multicolumn{1} { > {\centering}V{3.300000 in}} { \textbf{value}} & 
 \multicolumn{1} { > {\centering} V{3.000000 in} } {\textbf{description}} \\ \hline\hline \endfirsthead
            \multicolumn{2}{c}{Table \thetable\ sampling\_strategy (cont.)} \\
            % column headers on additional pages
            \hline\hline \multicolumn{1} { > {\centering}V{3.300000 in} } { \textbf{value}} & 
 \multicolumn{1} { > {\centering} V{3.000000 in} } {\textbf{description}} \\ \hline\hline \endhead
            % Footer on 1st to penultimate pages
            \multicolumn{2}{r}{{Continued on next page}} \\
            \endfoot
            % Footer on last page of table
            \hline
            \multicolumn{2}{r}{{End of table}} \\ 
            \endlastfoot
        }
    },
    %
    end table=\end{longtable},
    col sep=tab,
    string type,
    columns/value/.style={
            string type, 
            column type= V{3.300000 in}, 
            string replace*={_}{\_}
        },
    columns/description/.style={
            string type, 
            string replace*={_}{\_},
            column type = V{3.000000 in}
        }
    ]{/Users/dyb/GitHub/C3S_311a_CDM/tables/sampling_strategy.csv}

\pgfplotstabletypeset[
    empty header,
    begin table=\begin{longtable},
    %every head row/.style={output empty row},
    every nth row={1}{before row=\hline},
    every first row/.append style={
        before row={%
            % Initial caption
            \caption{sea\_level\_datum}
            \label{tab:DataTableSealeveldatum}\\
            % Initial column headers
            \hline\hline             \multicolumn{1} { V{4.300000 in}} { \textbf{\seqsplit{value}}} & 
            \multicolumn{1} { V{3.000000 in} } {\textbf{\seqsplit{description}}} \\ \hline\hline \endfirsthead
            \multicolumn{2}{c}{Table \thetable\ sea\_level\_datum (cont.)} \\
            % column headers on additional pages
            \hline\hline             \multicolumn{1} {V{4.300000 in} } { \textbf{\seqsplit{value}}} & 
            \multicolumn{1} { V{3.000000 in} } {\textbf{\seqsplit{description}}} \\ \hline\hline \endhead
            % Footer on 1st to penultimate pages
            \multicolumn{2}{r}{{Continued on next page}} \\
            \endfoot
            % Footer on last page of table
            \hline
            \multicolumn{2}{r}{{End of table}} \\ 
            \endlastfoot
        }
    },
    %
    end table=\end{longtable},
    col sep=tab,
    string type,
    columns/value/.style={
            string type, 
            column type= V{4.300000 in}, 
            string replace*={_}{\_}
        },
    columns/description/.style={
            string type, 
            string replace*={_}{\_},
            column type = V{3.000000 in}
        }
    ]{/Users/dyb/GitHub/C3S_311a_CDM/tables/sea_level_datum.csv}

\begin{landscape}
\pgfplotstabletypeset[
    empty header,
    begin table=\begin{longtable},
    %every head row/.style={output empty row},
    every nth row={1}{before row=\hline},
    every first row/.append style={
        before row={%
            % Initial caption
            \caption{sensor\_configuration\_fields}
            \label{tab:DataTableSensorconfigurationfields}\\
            % Initial column headers
            \hline\hline             \multicolumn{1} { V{1.180000 in}} { \textbf{\seqsplit{index}}} & 
            \multicolumn{1} { V{1.180000 in}} { \textbf{\seqsplit{field}}} & 
            \multicolumn{1} { V{1.180000 in}} { \textbf{\seqsplit{field\_name}}} & 
            \multicolumn{1} { V{1.180000 in}} { \textbf{\seqsplit{parameter}}} & 
            \multicolumn{1} { V{1.180000 in}} { \textbf{\seqsplit{type}}} & 
            \multicolumn{1} { V{3.000000 in} } {\textbf{\seqsplit{decription}}} \\ \hline\hline \endfirsthead
            \multicolumn{6}{c}{Table \thetable\ sensor\_configuration\_fields (cont.)} \\
            % column headers on additional pages
            \hline\hline             \multicolumn{1} {V{1.180000 in} } { \textbf{\seqsplit{index}}} & 
            \multicolumn{1} {V{1.180000 in} } { \textbf{\seqsplit{field}}} & 
            \multicolumn{1} {V{1.180000 in} } { \textbf{\seqsplit{field\_name}}} & 
            \multicolumn{1} {V{1.180000 in} } { \textbf{\seqsplit{parameter}}} & 
            \multicolumn{1} {V{1.180000 in} } { \textbf{\seqsplit{type}}} & 
            \multicolumn{1} { V{3.000000 in} } {\textbf{\seqsplit{decription}}} \\ \hline\hline \endhead
            % Footer on 1st to penultimate pages
            \multicolumn{6}{r}{{Continued on next page}} \\
            \endfoot
            % Footer on last page of table
            \hline
            \multicolumn{6}{r}{{End of table}} \\ 
            \endlastfoot
        }
    },
    %
    end table=\end{longtable},
    col sep=tab,
    string type,
    columns/index/.style={
            string type, 
            column type= V{1.180000 in}, 
            string replace*={_}{\_}
        },
    columns/field/.style={
            string type, 
            column type= V{1.180000 in}, 
            string replace*={_}{\_}
        },
    columns/field_name/.style={
            string type, 
            column type= n{1.180000 in}, 
            string replace*={_}{\_}
        },
    columns/parameter/.style={
            string type, 
            column type= V{1.180000 in}, 
            string replace*={_}{\_}
        },
    columns/type/.style={
            string type, 
            column type= V{1.180000 in}, 
            string replace*={_}{\_}
        },
    columns/decription/.style={
            string type, 
            string replace*={_}{\_},
            column type = V{3.000000 in}
        }
    ]{/Users/dyb/GitHub/C3S_311a_CDM/tables/sensor_configuration_fields.csv}
\end{landscape}

\begin{landscape}
\pgfplotstabletypeset[
    empty header,
    begin table=\begin{longtable},
    %every head row/.style={output empty row},
    every nth row={1}{before row=\hline},
    every first row/.append style={
        before row={%
            % Initial caption
            \caption{Source configuration fields}
            \label{tab:DataTable}\\
            % Initial column headers
            \hline\hline \multicolumn{1} { > {\centering}V{1.116667 in}} { \textbf{Value}} & 
\multicolumn{1} { > {\centering}V{1.116667 in}} { \textbf{Field}} & 
\multicolumn{1} { > {\centering}V{1.116667 in}} { \textbf{FieldName}} & 
\multicolumn{1} { > {\centering}V{1.116667 in}} { \textbf{Kind}} & 
\multicolumn{1} { > {\centering}V{1.116667 in}} { \textbf{CodeValue}} & 
\multicolumn{1} { > {\centering}V{1.116667 in}} { \textbf{Description}} & 
 \multicolumn{1} { > {\centering} V{3.000000 in} } {\textbf{Extended Description}} \\ \hline\hline \endfirsthead
            \multicolumn{7}{c}{Table \thetable\ Source configuration fields (cont.)} \\
            % column headers on additional pages
            \hline\hline \multicolumn{1} { > {\centering}V{1.116667 in} } { \textbf{Value}} & 
\multicolumn{1} { > {\centering}V{1.116667 in} } { \textbf{Field}} & 
\multicolumn{1} { > {\centering}V{1.116667 in} } { \textbf{FieldName}} & 
\multicolumn{1} { > {\centering}V{1.116667 in} } { \textbf{Kind}} & 
\multicolumn{1} { > {\centering}V{1.116667 in} } { \textbf{CodeValue}} & 
\multicolumn{1} { > {\centering}V{1.116667 in} } { \textbf{Description}} & 
 \multicolumn{1} { > {\centering} V{3.000000 in} } {\textbf{Extended Description}} \\ \hline\hline \endhead
            % Footer on 1st to penultimate pages
            \multicolumn{7}{r}{{Continued on next page}} \\
            \endfoot
            % Footer on last page of table
            \hline
            \multicolumn{7}{r}{{End of table}} \\ 
            \endlastfoot
        }
    },
    %
    end table=\end{longtable},
    col sep=tab,
    string type,
    columns/Value/.style={
            string type, 
            column type= V{1.116667 in}, 
            string replace*={_}{}
        },
    columns/Field/.style={
            string type, 
            column type= V{1.116667 in}, 
            string replace*={_}{}
        },
    columns/FieldName/.style={
            string type, 
            column type= V{1.116667 in}, 
            string replace*={_}{}
        },
    columns/Kind/.style={
            string type, 
            column type= V{1.116667 in}, 
            string replace*={_}{}
        },
    columns/CodeValue/.style={
            string type, 
            column type= V{1.116667 in}, 
            string replace*={_}{}
        },
    columns/Description/.style={
            string type, 
            column type= V{1.116667 in}, 
            string replace*={_}{}
        },
    columns/Extended Description/.style={
            string type, 
            string replace*={_}{},
            column type = V{3.000000 in}
        }
    ]{/Users/dyb/GitHub/C3S_311a_CDM/tables/source_configuration_fields.csv}
\end{landscape}

\pgfplotstabletypeset[
    empty header,
    begin table=\begin{longtable},
    every head row/.style={output empty row},
    every nth row={1}{before row=\hline},
    every first row/.append style={
        before row={%
            % Initial caption
            \caption{Source format}
            \label{tab:DataTable}\\
            % Initial column headers
            \hline\hline \multicolumn{1} { > {\centering}V{2 in}} { \textbf{Value}} & \multicolumn{1} { > {\centering} V{4 in} } {\textbf{Description}} \\ \hline\hline \endfirsthead
            \multicolumn{2}{c}{Table \thetable\ Source format (cont.)} \\
            % column headers on additional pages
            \hline\hline \multicolumn{1} { > {\centering}V{2 in} } { \textbf{Value}} & \multicolumn{1} { > {\centering} V{4 in} } {\textbf{Description}} \\ \hline\hline \endhead
            % Footer on 1st to penultimate pages
            \multicolumn{2}{r}{{Continued on next page}} \\
            \endfoot
            % Footer on last page of table
            \hline
            \multicolumn{2}{r}{{End of table}} \\ 
            \endlastfoot
        }
    },
    %
    end table=\end{longtable},
    col sep=tab,
    string type,
    columns/Value/.style={
            string type, 
            column type= V{2 in}, 
            string replace*={_}{}
        },
    columns/Description/.style={
            string type, 
            string replace*={_}{},
            column type = V{4 in}
        }
    ]{/Users/dyb/GitHub/C3S_311a_CDM/tables/source_format.csv}

\pgfplotstabletypeset[
    empty header,
    begin table=\begin{longtable},
    %every head row/.style={output empty row},
    every nth row={1}{before row=\hline},
    every first row/.append style={
        before row={%
            % Initial caption
            \caption{spatial\_representativeness}
            \label{tab:DataTableSpatialrepresentativeness}\\
            % Initial column headers
            \hline\hline             \multicolumn{1} { V{3.800000 in}} { \textbf{\seqsplit{value}}} & 
            \multicolumn{1} { V{3.000000 in} } {\textbf{\seqsplit{description}}} \\ \hline\hline \endfirsthead
            \multicolumn{2}{c}{Table \thetable\ spatial\_representativeness (cont.)} \\
            % column headers on additional pages
            \hline\hline             \multicolumn{1} {V{3.800000 in} } { \textbf{\seqsplit{value}}} & 
            \multicolumn{1} { V{3.000000 in} } {\textbf{\seqsplit{description}}} \\ \hline\hline \endhead
            % Footer on 1st to penultimate pages
            \multicolumn{2}{r}{{Continued on next page}} \\
            \endfoot
            % Footer on last page of table
            \hline
            \multicolumn{2}{r}{{End of table}} \\ 
            \endlastfoot
        }
    },
    %
    end table=\end{longtable},
    col sep=tab,
    string type,
    columns/value/.style={
            string type, 
            column type= V{3.800000 in}, 
            string replace*={_}{\_}
        },
    columns/description/.style={
            string type, 
            string replace*={_}{\_},
            column type = V{3.000000 in}
        }
    ]{/Users/dyb/GitHub/C3S_311a_CDM/tables/spatial_representativeness.csv}

\begin{landscape}
\pgfplotstabletypeset[
    empty header,
    begin table=\begin{longtable},
    %every head row/.style={output empty row},
    every nth row={1}{before row=\hline},
    every first row/.append style={
        before row={%
            % Initial caption
            \caption{station\_configuration\_fields}
            \label{tab:DataTableStationconfigurationfields}\\
            % Initial column headers
            \hline\hline             \multicolumn{1} { V{1.475000 in}} { \textbf{\seqsplit{index}}} & 
            \multicolumn{1} { V{1.475000 in}} { \textbf{\seqsplit{field}}} & 
            \multicolumn{1} { V{1.475000 in}} { \textbf{\seqsplit{field\_name}}} & 
            \multicolumn{1} { V{1.475000 in}} { \textbf{\seqsplit{kind}}} & 
            \multicolumn{1} { V{3.000000 in} } {\textbf{\seqsplit{description}}} \\ \hline\hline \endfirsthead
            \multicolumn{5}{c}{Table \thetable\ station\_configuration\_fields (cont.)} \\
            % column headers on additional pages
            \hline\hline             \multicolumn{1} {V{1.475000 in} } { \textbf{\seqsplit{index}}} & 
            \multicolumn{1} {V{1.475000 in} } { \textbf{\seqsplit{field}}} & 
            \multicolumn{1} {V{1.475000 in} } { \textbf{\seqsplit{field\_name}}} & 
            \multicolumn{1} {V{1.475000 in} } { \textbf{\seqsplit{kind}}} & 
            \multicolumn{1} { V{3.000000 in} } {\textbf{\seqsplit{description}}} \\ \hline\hline \endhead
            % Footer on 1st to penultimate pages
            \multicolumn{5}{r}{{Continued on next page}} \\
            \endfoot
            % Footer on last page of table
            \hline
            \multicolumn{5}{r}{{End of table}} \\ 
            \endlastfoot
        }
    },
    %
    end table=\end{longtable},
    col sep=tab,
    string type,
    columns/index/.style={
            string type, 
            column type= V{1.475000 in}, 
            string replace*={_}{\_}
        },
    columns/field/.style={
            string type, 
            column type= V{1.475000 in}, 
            string replace*={_}{\_}
        },
    columns/field_name/.style={
            string type, 
            column type= n{1.475000 in}, 
            string replace*={_}{\_}
        },
    columns/kind/.style={
            string type, 
            column type= V{1.475000 in}, 
            string replace*={_}{\_}
        },
    columns/description/.style={
            string type, 
            string replace*={_}{\_},
            column type = V{3.000000 in}
        }
    ]{/Users/dyb/GitHub/C3S_311a_CDM/tables/station_configuration_fields.csv}
\end{landscape}

\pgfplotstabletypeset[
    empty header,
    begin table=\begin{longtable},
    %every head row/.style={output empty row},
    every nth row={1}{before row=\hline},
    every first row/.append style={
        before row={%
            % Initial caption
            \caption{Station type}
            \label{tab:DataTable}\\
            % Initial column headers
            \hline\hline \multicolumn{1} { > {\centering}V{3.300000 in}} { \textbf{Value}} & 
 \multicolumn{1} { > {\centering} V{3.000000 in} } {\textbf{Description}} \\ \hline\hline \endfirsthead
            \multicolumn{2}{c}{Table \thetable\ Station type (cont.)} \\
            % column headers on additional pages
            \hline\hline \multicolumn{1} { > {\centering}V{3.300000 in} } { \textbf{Value}} & 
 \multicolumn{1} { > {\centering} V{3.000000 in} } {\textbf{Description}} \\ \hline\hline \endhead
            % Footer on 1st to penultimate pages
            \multicolumn{2}{r}{{Continued on next page}} \\
            \endfoot
            % Footer on last page of table
            \hline
            \multicolumn{2}{r}{{End of table}} \\ 
            \endlastfoot
        }
    },
    %
    end table=\end{longtable},
    col sep=tab,
    string type,
    columns/Value/.style={
            string type, 
            column type= V{3.300000 in}, 
            string replace*={_}{}
        },
    columns/Description/.style={
            string type, 
            string replace*={_}{},
            column type = V{3.000000 in}
        }
    ]{/Users/dyb/GitHub/C3S_311a_CDM/tables/station_type.csv}

\pgfplotstabletypeset[
    empty header,
    begin table=\begin{longtable},
    %every head row/.style={output empty row},
    every nth row={1}{before row=\hline},
    every first row/.append style={
        before row={%
            % Initial caption
            \caption{Sub region}
            \label{tab:DataTable}\\
            % Initial column headers
            \hline\hline \multicolumn{1} { > {\centering}V{1.100000 in}} { \textbf{Value}} & 
\multicolumn{1} { > {\centering}V{1.100000 in}} { \textbf{Type}} & 
\multicolumn{1} { > {\centering}V{1.100000 in}} { \textbf{Code}} & 
 \multicolumn{1} { > {\centering} V{3.000000 in} } {\textbf{Subregion}} \\ \hline\hline \endfirsthead
            \multicolumn{4}{c}{Table \thetable\ Sub region (cont.)} \\
            % column headers on additional pages
            \hline\hline \multicolumn{1} { > {\centering}V{1.100000 in} } { \textbf{Value}} & 
\multicolumn{1} { > {\centering}V{1.100000 in} } { \textbf{Type}} & 
\multicolumn{1} { > {\centering}V{1.100000 in} } { \textbf{Code}} & 
 \multicolumn{1} { > {\centering} V{3.000000 in} } {\textbf{Subregion}} \\ \hline\hline \endhead
            % Footer on 1st to penultimate pages
            \multicolumn{4}{r}{{Continued on next page}} \\
            \endfoot
            % Footer on last page of table
            \hline
            \multicolumn{4}{r}{{End of table}} \\ 
            \endlastfoot
        }
    },
    %
    end table=\end{longtable},
    col sep=tab,
    string type,
    columns/Value/.style={
            string type, 
            column type= V{1.100000 in}, 
            string replace*={_}{}
        },
    columns/Type/.style={
            string type, 
            column type= V{1.100000 in}, 
            string replace*={_}{}
        },
    columns/Code/.style={
            string type, 
            column type= V{1.100000 in}, 
            string replace*={_}{}
        },
    columns/Subregion/.style={
            string type, 
            string replace*={_}{},
            column type = V{3.000000 in}
        }
    ]{/Users/dyb/GitHub/C3S_311a_CDM/tables/sub_region.csv}

\pgfplotstabletypeset[
    empty header,
    begin table=\begin{longtable},
    every head row/.style={output empty row},
    every nth row={1}{before row=\hline},
    every first row/.append style={
        before row={%
            % Initial caption
            \caption{Time quality}
            \label{tab:DataTable}\\
            % Initial column headers
            \hline\hline \multicolumn{1} { > {\centering}V{2 in}} { \textbf{Value}} & \multicolumn{1} { > {\centering} V{4 in} } {\textbf{Description}} \\ \hline\hline \endfirsthead
            \multicolumn{2}{c}{Table \thetable\ Time quality (cont.)} \\
            % column headers on additional pages
            \hline\hline \multicolumn{1} { > {\centering}V{2 in} } { \textbf{Value}} & \multicolumn{1} { > {\centering} V{4 in} } {\textbf{Description}} \\ \hline\hline \endhead
            % Footer on 1st to penultimate pages
            \multicolumn{2}{r}{{Continued on next page}} \\
            \endfoot
            % Footer on last page of table
            \hline
            \multicolumn{2}{r}{{End of table}} \\ 
            \endlastfoot
        }
    },
    %
    end table=\end{longtable},
    col sep=tab,
    string type,
    columns/Value/.style={
            string type, 
            column type= V{2 in}, 
            string replace*={_}{}
        },
    columns/Description/.style={
            string type, 
            string replace*={_}{},
            column type = V{4 in}
        }
    ]{/Users/dyb/GitHub/C3S_311a_CDM/tables/time_quality.csv}

\pgfplotstabletypeset[
    empty header,
    begin table=\begin{longtable},
    %every head row/.style={output empty row},
    every nth row={1}{before row=\hline},
    every first row/.append style={
        before row={%
            % Initial caption
            \caption{time\_reference}
            \label{tab:DataTableTimereference}\\
            % Initial column headers
            \hline\hline             \multicolumn{1} { V{1.900000 in}} { \textbf{\seqsplit{index}}} & 
            \multicolumn{1} { V{1.900000 in}} { \textbf{\seqsplit{time\_reference}}} & 
            \multicolumn{1} { V{3.000000 in} } {\textbf{\seqsplit{description}}} \\ \hline\hline \endfirsthead
            \multicolumn{3}{c}{Table \thetable\ time\_reference (cont.)} \\
            % column headers on additional pages
            \hline\hline             \multicolumn{1} {V{1.900000 in} } { \textbf{\seqsplit{index}}} & 
            \multicolumn{1} {V{1.900000 in} } { \textbf{\seqsplit{time\_reference}}} & 
            \multicolumn{1} { V{3.000000 in} } {\textbf{\seqsplit{description}}} \\ \hline\hline \endhead
            % Footer on 1st to penultimate pages
            \multicolumn{3}{r}{{Continued on next page}} \\
            \endfoot
            % Footer on last page of table
            \hline
            \multicolumn{3}{r}{{End of table}} \\ 
            \endlastfoot
        }
    },
    %
    end table=\end{longtable},
    col sep=tab,
    string type,
    columns/index/.style={
            string type, 
            column type= V{1.900000 in}, 
            string replace*={_}{\_}
        },
    columns/time_reference/.style={
            string type, 
            column type= V{1.900000 in}, 
            string replace*={_}{\_}
        },
    columns/description/.style={
            string type, 
            string replace*={_}{\_},
            column type = V{3.000000 in}
        }
    ]{/Users/dyb/GitHub/C3S_311a_CDM/tables/time_reference.csv}

\pgfplotstabletypeset[
    empty header,
    begin table=\begin{longtable},
    %every head row/.style={output empty row},
    every nth row={1}{before row=\hline},
    every first row/.append style={
        before row={%
            % Initial caption
            \caption{traceability}
            \label{tab:DataTableTraceability}\\
            % Initial column headers
            \hline\hline \multicolumn{1} { > {\centering}V{3.300000 in}} { \textbf{value}} & 
 \multicolumn{1} { > {\centering} V{3.000000 in} } {\textbf{description}} \\ \hline\hline \endfirsthead
            \multicolumn{2}{c}{Table \thetable\ traceability (cont.)} \\
            % column headers on additional pages
            \hline\hline \multicolumn{1} { > {\centering}V{3.300000 in} } { \textbf{value}} & 
 \multicolumn{1} { > {\centering} V{3.000000 in} } {\textbf{description}} \\ \hline\hline \endhead
            % Footer on 1st to penultimate pages
            \multicolumn{2}{r}{{Continued on next page}} \\
            \endfoot
            % Footer on last page of table
            \hline
            \multicolumn{2}{r}{{End of table}} \\ 
            \endlastfoot
        }
    },
    %
    end table=\end{longtable},
    col sep=tab,
    string type,
    columns/value/.style={
            string type, 
            column type= V{3.300000 in}, 
            string replace*={_}{\_}
        },
    columns/description/.style={
            string type, 
            string replace*={_}{\_},
            column type = V{3.000000 in}
        }
    ]{/Users/dyb/GitHub/C3S_311a_CDM/tables/traceability.csv}

\begin{landscape}
\pgfplotstabletypeset[
    empty header,
    begin table=\begin{longtable},
    %every head row/.style={output empty row},
    every nth row={1}{before row=\hline},
    every first row/.append style={
        before row={%
            % Initial caption
            \caption{units}
            \label{tab:DataTableUnits}\\
            % Initial column headers
            \hline\hline             \multicolumn{1} { V{1.540000 in}} { \textbf{\seqsplit{value}}} & 
            \multicolumn{1} { V{1.540000 in}} { \textbf{\seqsplit{units}}} & 
            \multicolumn{1} { V{1.540000 in}} { \textbf{\seqsplit{conventional\_abbreviation}}} & 
            \multicolumn{1} { V{1.540000 in}} { \textbf{\seqsplit{abbreviation\_in\_ASCII}}} & 
            \multicolumn{1} { V{1.540000 in}} { \textbf{\seqsplit{abbreviation\_in\_ITA2}}} & 
            \multicolumn{1} { V{3.000000 in} } {\textbf{\seqsplit{definition\_in\_base\_units}}} \\ \hline\hline \endfirsthead
            \multicolumn{6}{c}{Table \thetable\ units (cont.)} \\
            % column headers on additional pages
            \hline\hline             \multicolumn{1} {V{1.540000 in} } { \textbf{\seqsplit{value}}} & 
            \multicolumn{1} {V{1.540000 in} } { \textbf{\seqsplit{units}}} & 
            \multicolumn{1} {V{1.540000 in} } { \textbf{\seqsplit{conventional\_abbreviation}}} & 
            \multicolumn{1} {V{1.540000 in} } { \textbf{\seqsplit{abbreviation\_in\_ASCII}}} & 
            \multicolumn{1} {V{1.540000 in} } { \textbf{\seqsplit{abbreviation\_in\_ITA2}}} & 
            \multicolumn{1} { V{3.000000 in} } {\textbf{\seqsplit{definition\_in\_base\_units}}} \\ \hline\hline \endhead
            % Footer on 1st to penultimate pages
            \multicolumn{6}{r}{{Continued on next page}} \\
            \endfoot
            % Footer on last page of table
            \hline
            \multicolumn{6}{r}{{End of table}} \\ 
            \endlastfoot
        }
    },
    %
    end table=\end{longtable},
    col sep=tab,
    string type,
    columns/value/.style={
            string type, 
            column type= V{1.540000 in}, 
            string replace*={_}{\_}
        },
    columns/units/.style={
            string type, 
            column type= V{1.540000 in}, 
            string replace*={_}{\_}
        },
    columns/conventional_abbreviation/.style={
            string type, 
            column type= V{1.540000 in}, 
            string replace*={_}{\_}
        },
    columns/abbreviation_in_ASCII/.style={
            string type, 
            column type= V{1.540000 in}, 
            string replace*={_}{\_}
        },
    columns/abbreviation_in_ITA2/.style={
            string type, 
            column type= V{1.540000 in}, 
            string replace*={_}{\_}
        },
    columns/definition_in_base_units/.style={
            string type, 
            string replace*={_}{\_},
            column type = V{3.000000 in}
        }
    ]{/Users/dyb/GitHub/C3S_311a_CDM/tables/units.csv}
\end{landscape}

\pgfplotstabletypeset[
    empty header,
    begin table=\begin{longtable},
    %every head row/.style={output empty row},
    every nth row={1}{before row=\hline},
    every first row/.append style={
        before row={%
            % Initial caption
            \caption{update\_frequency}
            \label{tab:DataTableUpdatefrequency}\\
            % Initial column headers
            \hline\hline \multicolumn{1} { > {\centering}V{3.300000 in}} { \textbf{value}} & 
 \multicolumn{1} { > {\centering} V{3.000000 in} } {\textbf{description}} \\ \hline\hline \endfirsthead
            \multicolumn{2}{c}{Table \thetable\ update\_frequency (cont.)} \\
            % column headers on additional pages
            \hline\hline \multicolumn{1} { > {\centering}V{3.300000 in} } { \textbf{value}} & 
 \multicolumn{1} { > {\centering} V{3.000000 in} } {\textbf{description}} \\ \hline\hline \endhead
            % Footer on 1st to penultimate pages
            \multicolumn{2}{r}{{Continued on next page}} \\
            \endfoot
            % Footer on last page of table
            \hline
            \multicolumn{2}{r}{{End of table}} \\ 
            \endlastfoot
        }
    },
    %
    end table=\end{longtable},
    col sep=tab,
    string type,
    columns/value/.style={
            string type, 
            column type= V{3.300000 in}, 
            string replace*={_}{\_}
        },
    columns/description/.style={
            string type, 
            string replace*={_}{\_},
            column type = V{3.000000 in}
        }
    ]{/Users/dyb/GitHub/C3S_311a_CDM/tables/update_frequency.csv}

\pgfplotstabletypeset[
    empty header,
    begin table=\begin{longtable},
    every head row/.style={output empty row},
    every nth row={1}{before row=\hline},
    every first row/.append style={
        before row={%
            % Initial caption
            \caption{Z coordinate method}
            \label{tab:DataTable}\\
            % Initial column headers

\pgfplotstabletypeset[
    empty header,
    begin table=\begin{longtable},
    %every head row/.style={output empty row},
    every nth row={1}{before row=\hline},
    every first row/.append style={
        before row={%
            % Initial caption
            \caption{z\_coordinate\_type}
            \label{tab:DataTableZcoordinatetype}\\
            % Initial column headers
            \hline\hline             \multicolumn{1} { V{4.300000 in}} { \textbf{\seqsplit{value}}} & 
            \multicolumn{1} { V{3.000000 in} } {\textbf{\seqsplit{description}}} \\ \hline\hline \endfirsthead
            \multicolumn{2}{c}{Table \thetable\ z\_coordinate\_type (cont.)} \\
            % column headers on additional pages
            \hline\hline             \multicolumn{1} {V{4.300000 in} } { \textbf{\seqsplit{value}}} & 
            \multicolumn{1} { V{3.000000 in} } {\textbf{\seqsplit{description}}} \\ \hline\hline \endhead
            % Footer on 1st to penultimate pages
            \multicolumn{2}{r}{{Continued on next page}} \\
            \endfoot
            % Footer on last page of table
            \hline
            \multicolumn{2}{r}{{End of table}} \\ 
            \endlastfoot
        }
    },
    %
    end table=\end{longtable},
    col sep=tab,
    string type,
    columns/value/.style={
            string type, 
            column type= V{4.300000 in}, 
            string replace*={_}{\_}
        },
    columns/description/.style={
            string type, 
            string replace*={_}{\_},
            column type = V{3.000000 in}
        }
    ]{/Users/dyb/GitHub/C3S_311a_CDM/tables/z_coordinate_type.csv}


%\include {surface_type}
% \include {surface_type_scheme}
% \incude {topography}


%\begin{landscape}
\pgfplotstabletypeset[
    empty header,
    begin table=\begin{longtable},
    %every head row/.style={output empty row},
    every nth row={1}{before row=\hline},
    every first row/.append style={
        before row={%
            % Initial caption
            \caption{observation\_code\_tables}
            \label{tab:DataTableObservationcodetables}\\
            % Initial column headers
            \hline\hline             \multicolumn{1} { V{1.180000 in}} { \textbf{\seqsplit{index}}} & 
            \multicolumn{1} { V{1.180000 in}} { \textbf{\seqsplit{code\_table\_scheme}}} & 
            \multicolumn{1} { V{1.180000 in}} { \textbf{\seqsplit{code\_table\_id}}} & 
            \multicolumn{1} { V{1.180000 in}} { \textbf{\seqsplit{code\_table\_name}}} & 
            \multicolumn{1} { V{1.180000 in}} { \textbf{\seqsplit{value}}} & 
            \multicolumn{1} { V{3.000000 in} } {\textbf{\seqsplit{description}}} \\ \hline\hline \endfirsthead
            \multicolumn{6}{c}{Table \thetable\ observation\_code\_tables (cont.)} \\
            % column headers on additional pages
            \hline\hline             \multicolumn{1} {V{1.180000 in} } { \textbf{\seqsplit{index}}} & 
            \multicolumn{1} {V{1.180000 in} } { \textbf{\seqsplit{code\_table\_scheme}}} & 
            \multicolumn{1} {V{1.180000 in} } { \textbf{\seqsplit{code\_table\_id}}} & 
            \multicolumn{1} {V{1.180000 in} } { \textbf{\seqsplit{code\_table\_name}}} & 
            \multicolumn{1} {V{1.180000 in} } { \textbf{\seqsplit{value}}} & 
            \multicolumn{1} { V{3.000000 in} } {\textbf{\seqsplit{description}}} \\ \hline\hline \endhead
            % Footer on 1st to penultimate pages
            \multicolumn{6}{r}{{Continued on next page}} \\
            \endfoot
            % Footer on last page of table
            \hline
            \multicolumn{6}{r}{{End of table}} \\ 
            \endlastfoot
        }
    },
    %
    end table=\end{longtable},
    col sep=tab,
    string type,
    columns/index/.style={
            string type, 
            column type= V{1.180000 in}, 
            string replace*={_}{\_}
        },
    columns/code_table_scheme/.style={
            string type, 
            column type= V{1.180000 in}, 
            string replace*={_}{\_}
        },
    columns/code_table_id/.style={
            string type, 
            column type= V{1.180000 in}, 
            string replace*={_}{\_}
        },
    columns/code_table_name/.style={
            string type, 
            column type= n{1.180000 in}, 
            string replace*={_}{\_}
        },
    columns/value/.style={
            string type, 
            column type= V{1.180000 in}, 
            string replace*={_}{\_}
        },
    columns/description/.style={
            string type, 
            string replace*={_}{\_},
            column type = V{3.000000 in}
        }
    ]{/Users/dyb/GitHub/C3S_311a_CDM/tables/observation_code_tables.csv}
\end{landscape}

%\pgfplotstabletypeset[
    empty header,
    begin table=\begin{longtable},
    %every head row/.style={output empty row},
    every nth row={1}{before row=\hline},
    every first row/.append style={
        before row={%
            % Initial caption
            \caption{Method of estimating uncertainty}
            \label{tab:DataTable}\\
            % Initial column headers
            \hline\hline \multicolumn{1} { > {\centering}V{3.300000 in}} { \textbf{Value}} & 
 \multicolumn{1} { > {\centering} V{3.000000 in} } {\textbf{Description}} \\ \hline\hline \endfirsthead
            \multicolumn{2}{c}{Table \thetable\ Method of estimating uncertainty (cont.)} \\
            % column headers on additional pages
            \hline\hline \multicolumn{1} { > {\centering}V{3.300000 in} } { \textbf{Value}} & 
 \multicolumn{1} { > {\centering} V{3.000000 in} } {\textbf{Description}} \\ \hline\hline \endhead
            % Footer on 1st to penultimate pages
            \multicolumn{2}{r}{{Continued on next page}} \\
            \endfoot
            % Footer on last page of table
            \hline
            \multicolumn{2}{r}{{End of table}} \\ 
            \endlastfoot
        }
    },
    %
    end table=\end{longtable},
    col sep=tab,
    string type,
    columns/Value/.style={
            string type, 
            column type= V{3.300000 in}, 
            string replace*={_}{}
        },
    columns/Description/.style={
            string type, 
            string replace*={_}{},
            column type = V{3.000000 in}
        }
    ]{/Users/dyb/GitHub/C3S_311a_CDM/tables/method_of_estimating_uncertainty.csv}

%\pgfplotstabletypeset[
    empty header,
    begin table=\begin{longtable},
    %every head row/.style={output empty row},
    every nth row={1}{before row=\hline},
    every first row/.append style={
        before row={%
            % Initial caption
            \caption{Processing code}
            \label{tab:DataTable}\\
            % Initial column headers
            \hline\hline \multicolumn{1} { > {\centering}V{3.300000 in}} { \textbf{Value}} & 
 \multicolumn{1} { > {\centering} V{3.000000 in} } {\textbf{Description}} \\ \hline\hline \endfirsthead
            \multicolumn{2}{c}{Table \thetable\ Processing code (cont.)} \\
            % column headers on additional pages
            \hline\hline \multicolumn{1} { > {\centering}V{3.300000 in} } { \textbf{Value}} & 
 \multicolumn{1} { > {\centering} V{3.000000 in} } {\textbf{Description}} \\ \hline\hline \endhead
            % Footer on 1st to penultimate pages
            \multicolumn{2}{r}{{Continued on next page}} \\
            \endfoot
            % Footer on last page of table
            \hline
            \multicolumn{2}{r}{{End of table}} \\ 
            \endlastfoot
        }
    },
    %
    end table=\end{longtable},
    col sep=tab,
    string type,
    columns/Value/.style={
            string type, 
            column type= V{3.300000 in}, 
            string replace*={_}{}
        },
    columns/Description/.style={
            string type, 
            string replace*={_}{},
            column type = V{3.000000 in}
        }
    ]{/Users/dyb/Documents/Projects/C3S_311a_Lot2/WP2/CDM/github/tables/processing_code.csv}

%\documentclass{article}

\usepackage[a4paper, total={6in, 9.5in}]{geometry}
\usepackage{pgfplotstable} 
\usepackage{longtable}
\usepackage{booktabs}
\renewcommand{\familydefault}{\sfdefault}




\pgfplotstableset{
begin table=\begin{longtable},
end table=\end{longtable},
}


\title {Copernicus Climate Change Service - 311a Lot 2\\Defining a common data model}
\author {David I. Berry}
\date {23 June 2017}
\begin{document}
\maketitle
\section {Decode tables}

\begin{table}[h!]
  \begin{center}
    \caption{Measuring system model}
    \pgfplotstabletypeset[
        col sep=tab, string type,
        header=true,
        every head row/.style={before row=\toprule\toprule, after row=\bottomrule\bottomrule\endhead}, 
        every last row/.style={after row=\bottomrule},
        after row=\hline,
        columns={Value,Description},
        columns/Value/.style={string type, column type=l, string replace*={_}{}},
        columns/Description/.style={string type, column type=p{0.5\textwidth}, string replace*={_}{}}
    ]{/Users/dyb/Documents/Projects/C3S_311a_Lot2/WP2/CDM/github/tables/measuring_system_model.csv}
  \end{center}
\end{table}

\end{document}

%\pgfplotstabletypeset[
    empty header,
    begin table=\begin{longtable},
    %every head row/.style={output empty row},
    every nth row={1}{before row=\hline},
    every first row/.append style={
        before row={%
            % Initial caption
            \caption{product\_level}
            \label{tab:DataTableProductlevel}\\
            % Initial column headers
            \hline\hline             \multicolumn{1} { V{1.900000 in}} { \textbf{\seqsplit{index}}} & 
            \multicolumn{1} { V{1.900000 in}} { \textbf{\seqsplit{product\_level}}} & 
            \multicolumn{1} { V{3.000000 in} } {\textbf{\seqsplit{description}}} \\ \hline\hline \endfirsthead
            \multicolumn{3}{c}{Table \thetable\ product\_level (cont.)} \\
            % column headers on additional pages
            \hline\hline             \multicolumn{1} {V{1.900000 in} } { \textbf{\seqsplit{index}}} & 
            \multicolumn{1} {V{1.900000 in} } { \textbf{\seqsplit{product\_level}}} & 
            \multicolumn{1} { V{3.000000 in} } {\textbf{\seqsplit{description}}} \\ \hline\hline \endhead
            % Footer on 1st to penultimate pages
            \multicolumn{3}{r}{{Continued on next page}} \\
            \endfoot
            % Footer on last page of table
            \hline
            \multicolumn{3}{r}{{End of table}} \\ 
            \endlastfoot
        }
    },
    %
    end table=\end{longtable},
    col sep=tab,
    string type,
    columns/index/.style={
            string type, 
            column type= V{1.900000 in}, 
            string replace*={_}{\_}
        },
    columns/product_level/.style={
            string type, 
            column type= V{1.900000 in}, 
            string replace*={_}{\_}
        },
    columns/description/.style={
            string type, 
            string replace*={_}{\_},
            column type = V{3.000000 in}
        }
    ]{/Users/dyb/GitHub/C3S_311a_CDM/tables/product_level.csv}

%\pgfplotstabletypeset[
    empty header,
    begin table=\begin{longtable},
    %every head row/.style={output empty row},
    every nth row={1}{before row=\hline},
    every first row/.append style={
        before row={%
            % Initial caption
            \caption{product\_status}
            \label{tab:DataTableProductstatus}\\
            % Initial column headers
            \hline\hline             \multicolumn{1} { V{1.266667 in}} { \textbf{\seqsplit{index}}} & 
            \multicolumn{1} { V{1.266667 in}} { \textbf{\seqsplit{product\_status}}} & 
            \multicolumn{1} { V{1.266667 in}} { \textbf{\seqsplit{description}}} & 
            \multicolumn{1} { V{3.000000 in} } {\textbf{\seqsplit{extended\_description}}} \\ \hline\hline \endfirsthead
            \multicolumn{4}{c}{Table \thetable\ product\_status (cont.)} \\
            % column headers on additional pages
            \hline\hline             \multicolumn{1} {V{1.266667 in} } { \textbf{\seqsplit{index}}} & 
            \multicolumn{1} {V{1.266667 in} } { \textbf{\seqsplit{product\_status}}} & 
            \multicolumn{1} {V{1.266667 in} } { \textbf{\seqsplit{description}}} & 
            \multicolumn{1} { V{3.000000 in} } {\textbf{\seqsplit{extended\_description}}} \\ \hline\hline \endhead
            % Footer on 1st to penultimate pages
            \multicolumn{4}{r}{{Continued on next page}} \\
            \endfoot
            % Footer on last page of table
            \hline
            \multicolumn{4}{r}{{End of table}} \\ 
            \endlastfoot
        }
    },
    %
    end table=\end{longtable},
    col sep=tab,
    string type,
    columns/index/.style={
            string type, 
            column type= V{1.266667 in}, 
            string replace*={_}{\_}
        },
    columns/product_status/.style={
            string type, 
            column type= V{1.266667 in}, 
            string replace*={_}{\_}
        },
    columns/description/.style={
            string type, 
            column type= V{1.266667 in}, 
            string replace*={_}{\_}
        },
    columns/extended_description/.style={
            string type, 
            string replace*={_}{\_},
            column type = V{3.000000 in}
        }
    ]{/Users/dyb/GitHub/C3S_311a_CDM/tables/product_status.csv}










\end{document}

